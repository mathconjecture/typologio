\section{Αριθμητικές Μέθοδοι Σύνοψης Δεδομένων}
\twocolumnside{Μη Ομαλοποιημένα Δεδομένα}{Ομαλοποιημένα Δεδομένα}

\subsection{Μέτρα θέσης - Κεντρικής Τάσης}

\twocolumnside{\subsubsection{Αριθμητικός Μέσος}
\begin{align*}
    \overline{X} & =\frac{\sum_{i=1}^{n}X_{i}}{n} \\
    \overline{X} & \cong\frac{\sum_{i=1}^{k}f_{i}m_{i}}{n},\; n=\sum_{i=1}^{k}f_i
  \end{align*}}{\subsubsection{Σταθμικός Αριθμητικός Μέσος}
  \begin{equation*}
    \overline{X}_{w}=\frac{\sum_{i=1}^{n}w_{i}X_{i}}{\sum_{i=1}^{n}w_{i}}
  \end{equation*}}


\subsubsection{Διάμεσος}

\twocolumnside{
    Αν $n$ περιττός: $M$ η τιμή της παρατήρησης στη θέση $\frac{n}{2}+\frac{1}{2}$

   \begin{equation*}
   M=X_{\left(\frac{n}{2}+\frac{1}{2}\right)}
  \end{equation*}
    }{
    Αν $n$ άρτιος: 
    \begin{align*}
      M & =\frac{1}{2}\left(X_{\left(\frac{n}{2}+1\right)}+X_{\left(\frac{n}{2}\right)}\right)\\
      M & \cong L_{M}+\frac{\delta}{f_{M}}\left(\frac{n}{2}-F_{M-1}\right)
    \end{align*}
    }

\subsubsection{Επικρατούσα Τιμή}

$T_{0}$: Η τιμή με τη μεγαλύτερη συχνότητα εμφάνισης\hfill $T_{0}=L_{T_{0}}+\delta \frac{\Delta_{1}}{\Delta_{1}+\Delta_{2}}$


\subsubsection{$i$-Τεταρτημόριο}

Το $i=1,2,3$ τεταρτημόριο $(Q_{i})$ βρίσκεται στην $[\frac{i(n+1)}{4}]$ θέση. Η τιμή του $i=1,2,3$ τεταρτημορίου $(Q_{i})$ είναι \[Q_{i}=X_{(A_{Q})}+\Delta_{Q}[X_{(A_{Q}+1)}-X_{(A_{Q})}]\]όπου $A_{Q}$ είναι το ακέραιο μέρος του πηλίκου $[\frac{i(n+1)}{4}]$ και $\Delta_{Q}$ είναι το δεκαδικό μέρος του πηλίκου $[\frac{i(n+1)}{4}]$.

\[
Q_{i}=L_{Q_{i}}+\frac{\delta}{f_{Q_{i}}}\left(\frac{n\cdot i}{4}-F_{Q_{i}-1}\right)
\]

\subsection{Μέτρα Διασποράς}

\subsubsection{Εύρος}
\begin{multicols}{2}
  \[
  R=X_{max}-X_{min}
  \]
\end{multicols}


\subsubsection{Ενδοτεταρτημοριακό Εύρος}
\begin{multicols}{2}
  \[
  IR=Q_{3}-Q_{1}
  \]

  \[
  IR=Q_{3}-Q_{1}
  \]
\end{multicols}


\subsubsection{Τεταρτημοριακή Απόκλιση}

\begin{multicols}{2}
  \[
  Q=\frac{Q_{3}-Q_{1}}{2}
  \]

  \[
  Q=\frac{Q_{3}-Q_{1}}{2}
  \]

\end{multicols}

\subsubsection{Διακύμανση}

\begin{multicols}{2}
  \[
S_{\text{ορ}}^2=\frac{\sum_{i=1}^{n}(X_{i}-\overline{X})^{2}}{n}
  \]
  \[
S^{2}=\frac{\sum_{i=1}^{n}(X_{i}-\overline{X})^{2}}{n-1}
  \]
  \[
S^{2}=\frac{\sum_{i=1}^{n}X_{i}^{2}-n\overline{X}^{2}}{n-1}=\frac{\sum_{i=1}^{n}X_{i}^{2}}{n-1}-\frac{(\sum_{i=1}^{n}X_{i})^{2}}{n(n-1)}
  \]

  \columnbreak

\[
S^{2}\cong \frac{\sum_{i=1}^{k}f_{i}(m_{i}-\overline{X})^{2}}{n-1},\; n=\sum_{i=1}^{k}f_{i}
\]
\[
S^{2}\cong \frac{\sum_{i=1}^{k}f_{i}m_{i}^{2}-n\overline{X}^{2}}{n-1}=\frac{\sum_{i=1}^{k}f_{i}m_{i}^{2}}{n-1}-\frac{(\sum_{i=1}^{k}f_{i}m_{i})^{2}}{(n-1)n}
\]

\end{multicols}

\subsubsection{Τυπική Απόκλιση}

\begin{multicols}{2}
  \[
S_{\text{ορ}}=+\sqrt{S_{\text{ορ}}^{2}} \quad\text{ή}\quad S=+\sqrt{S^{2}}
  \]

  \[
S=+\sqrt{S^{2}}
  \]

\end{multicols}

\subsection{Μέτρα Σχετικής Μεταβλητότητας}


\subsubsection{Συντελεστής Μεταβλητότητας}

\begin{multicols}{2}

  \[
CV=\frac{S}{X}
  \]

  \[
CV=\frac{S}{X}
  \]

\end{multicols}

\subsection{Μέτρα Ασυμμετρίας}

\subsubsection{Συντελεστές Ασυμμετρίας}

\begin{multicols}{2}
  \[
S_{P}=\frac{\overline{X}-T_{0}}{S}
  \]

  \[
  \beta_{3}=\frac{\frac{\sum_{i=1}^{n}(X_{i}-\overline{X})^{3}}{n}}{S^{3}}
  \]

  \columnbreak

  \[
S_{P}=\frac{\overline{X}-T_{0}}{S}
  \]

  \[
\beta_{3}\frac{\frac{\sum_{i=1}^{k}f_{i}(m_{i}-\overline{X})^{3}}{n}}{S^{3}}
  \]

\end{multicols}


\subsection{Μέτρα Κύρτωσης}

\subsubsection{Συντελεστής Κύρτωσης}

\begin{multicols}{2}
  \[
  \beta_{4}=\frac{\frac{\sum_{i=1}^{n}(X_{i}-\overline{X})^{4}}{n}}{S^{4}}
  \]

  \[
\beta_{4}=\frac{\frac{\sum_{i=1}^{k}f_{i}(m_{i}-\overline{X})^{4}}{n}}{S^{4}}
  \]
\end{multicols}
