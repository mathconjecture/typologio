\documentclass[a4paper,10pt]{report}
\usepackage{etex}
%%%%%%%%%%%%%%%%%%%%%%%%%%%%%%%%%%%%%%
% Babel language package
\usepackage[english,greek]{babel}
% Inputenc font encoding
\usepackage[utf8]{inputenc}
%%%%%%%%%%%%%%%%%%%%%%%%%%%%%%%%%%%%%%

\usepackage{extsizes}
\usepackage{multicol}

%%%%% math packages %%%%%%%%%%%%%%%%%%
\usepackage[intlimits]{amsmath}
\usepackage{amssymb}
\usepackage{amsfonts}
\usepackage{amsthm}
\usepackage{proof}
\usepackage{mathtools}

\usepackage[italicdiff]{physics}
\usepackage{siunitx}
\usepackage{xfrac}

%%%%%%% symbols packages %%%%%%%%%%%%%%
\usepackage{bm} %for use \bm instead \boldsymbol in math mode
\usepackage{dsfont}
\usepackage{stmaryrd}
%%%%%%%%%%%%%%%%%%%%%%%%%%%%%%%%%%%%%%%


%%%%%% graphics %%%%%%%%%%%%%%%%%%%%%%%
\usepackage{graphicx}
\usepackage{color}
%\usepackage{xypic}
%\usepackage[all]{xy}
%\usepackage{calc}

%%%%%% tables %%%%%%%%%%%%%%%%%%%%%%%%%
\usepackage{array}
\usepackage{booktabs}
\usepackage{multirow}
\usepackage{makecell}
\usepackage{minibox}
\usepackage{systeme}
%%%%%%%%%%%%%%%%%%%%%%%%%%%%%%%%%%%%%%%

\usepackage{enumerate}

\usepackage{fancyhdr}
%%%%% header and footer rule %%%%%%%%%
\setlength{\headheight}{14pt}
\renewcommand{\headrulewidth}{0pt}
\renewcommand{\footrulewidth}{0pt}
\fancypagestyle{plain}{\fancyhf{}
\fancyhead{}
\lfoot{}
\rfoot{\small \thepage}}
\fancypagestyle{vangelis}{\fancyhf{}
\rhead{\small \leftmark}
\lhead{\small }
\lfoot{}
\rfoot{\small \thepage}}
%%%%%%%%%%%%%%%%%%%%%%%%%%%%%%%%%%%%%%%

%\usepackage{hyperref}
%\usepackage{url}
%%%%%%%% hyperref settings %%%%%%%%%%%%
%\hypersetup{pdfpagemode=UseOutlines,hidelinks,
%bookmarksopen=true,
%pdfdisplaydoctitle=true,
%pdfstartview=Fit,
%unicode=true,
%pdfpagelayout=OneColumn,
%}
%%%%%%%%%%%%%%%%%%%%%%%%%%%%%%%%%%%%%%%

\usepackage[space]{grffile}

\usepackage{geometry}
\geometry{left=25.63mm,right=25.63mm,top=36.25mm,bottom=36.25mm,footskip=24.16mm,headsep=24.16mm}

%\usepackage[explicit]{titlesec}
%%%%%% titlesec settings %%%%%%%%%%%%%
%\titleformat{\chapter}[block]{\LARGE\sc\bfseries}{\thechapter.}{1ex}{#1}
%\titlespacing*{\chapter}{0cm}{0cm}{36pt}[0ex]
%\titleformat{\section}[block]{\Large\bfseries}{\thesection.}{1ex}{#1}
%\titlespacing*{\section}{0cm}{34.56pt}{17.28pt}[0ex]
%\titleformat{\subsection}[block]{\large\bfseries{\thesubsection.}{1ex}{#1}
%\titlespacing*{\subsection}{0pt}{28.80pt}{14.40pt}[0ex]
%%%%%%%%%%%%%%%%%%%%%%%%%%%%%%%%%%%%%%

%%%%%%%%% My Theorems %%%%%%%%%%%%%%%%%%
\newtheorem{thm}{Θεώρημα}[section]
\newtheorem{cor}[thm]{Πόρισμα}
\newtheorem{lem}[thm]{λήμμα}
\theoremstyle{definition}
\newtheorem{dfn}{Ορισμός}[section]
\newtheorem{dfns}[dfn]{Ορισμοί}
\newtheorem{ex}[thm]{Παραδειγμα}
\theoremstyle{remark}
\newtheorem{remark}{Παρατήρηση}[section]
\newtheorem{remarks}[remark]{Παρατηρήσεις}
%%%%%%%%%%%%%%%%%%%%%%%%%%%%%%%%%%%%%%%

%%%%%%% nesting newcommands $$$$$$$$$$$$$$$$$$$
\newcommand{\function}[1]{\newcommand{\nvec}[2]{#1(##1_1,\ldots, ##1_##2)}}

\newcommand{\linode}[2]{#1_n(x)#2^{(n)}+#1_{n-1}(x)#2^{(n-1)}+\cdots +#1_0(x)#2=g(x)} 
\newcommand{\vecoffun}[3]{#1_0(#2),\ldots,#1_#3(#2)}

\newcommand{\mysum}[1]{\sum_{n=#1}^{\infty}}



\renewcommand{\vector}[1]{(x_1,x_2,\ldots,x_{#1})}
\newcommand{\avector}[2]{(#1_1,#1_2,\ldots,#1_{#2})}
\newcommand{\aDEFvector}[2][a]{(#1_1,#1_2,\ldots,#1_{#2})}

\newcommand{\rt}[3]{\vb{r}(t)=#1\,\vb{i}+#2\,\vb{j}+#3\,\vb{k}}
\newcommand{\rtt}[2]{\vb{r}(t)=#1\,\vb{i}+#2\,\vb{j}}
\newcommand{\rs}[3]{\vb{r}(s)=#1\,\vb{i}+#2\,\vb{j}+#3\,\vb{k}}
\newcommand{\rss}[2]{\vb{r}(s)=#1\,\vb{i}+#2\,\vb{j}}
\newcommand{\vect}[4]{\vb{#1}=#2\,\vb{i}+#3\,\vb{j}+#4\,\vb{k}}
\newcommand{\vectt}[3]{\vb{#1}=#2\,\vb{i}+#3\,\vb{j}}


\DeclareMathOperator{\Arg}{Arg}

%%%%%%%%% My Theorems %%%%%%%%%%%%%%%%%%
\newtheorem{thm}{Θεώρημα}[section]
\newtheorem{cor}[thm]{Πόρισμα}
\newtheorem{lem}[thm]{λήμμα}
\theoremstyle{definition}
\newtheorem{dfn}{Ορισμός}[section]
\newtheorem{dfns}[dfn]{Ορισμοί}
\newtheorem{ex}[thm]{Παραδειγμα}
\theoremstyle{remark}
\newtheorem{remark}{Παρατήρηση}[section]
\newtheorem{remarks}[remark]{Παρατηρήσεις}
%%%%%%%%%%%%%%%%%%%%%%%%%%%%%%%%%%%%%%%



\begin{document}

\section{Ορισμός των αριθμών \textlatin{Bernoulli}}

Οι αριθμοί \textlatin{Bernoulli} $ B_{1} $, $ B_{2} $, $ B_{3}, \ldots $   
ορίζονται από τις σειρές

\begin{tabular}{ll}
	$ \frac{x}{e^{x} - 1} = 1 - \frac{x}{2} + \frac{B_{1} x^{2}}{2!} -
	\frac{B_{2} x^{4}}{4!} + \frac{B_{3} x^{6}}{6!} - \cdots $ & $ \abs{x} < 2
	\pi $ \\
	$ 1 - \frac{x}{2} \cot{\frac{x}{2}} = \frac{B_{1} x^{2}}{2!} +
	\frac{B_{2} x^{4}}{4!} + \frac{B_{3} x^{6}}{6!} + \cdots $ & $ \abs{x} < \pi $
\end{tabular}

\section{Ορισμός των αριθμών \textlatin{Euler}}

Οι αριθμοί \textlatin{Euler} $ E_{1} $, $ E_{2} $, $ E_{3}, \ldots $ ορίζονται
από τις σειρές

\begin{tabular}{ll}
	$ \sech{x} = 1 - \frac{E_{1} x^{2}}{2!} + \frac{E_{2} x^{4}}{4!} -
	\frac{E_{3} x^{6}}{6!} + \cdots $ & $ \abs{x} < \frac{\pi}{2} $ \\
	$ \sec{x} = 1 + \frac{E_{1} x^{2}}{2!} + \frac{E_{2} x^{4}}{4!} -
	\frac{E_{3} x^{6}}{6!} + \cdots $ & $ \abs{x} < \frac{\pi}{2}  $
\end{tabular}


\section{Πίνακας μερικών αριθμών \textlatin{Bernoulli} και \textlatin{Euler}}

\begin{tabular}{ll}
	\toprule \\
	\textlatin{Bernoulli} & \textlatin{Euler} \\
	\midrule \\
	$ B_{1} = \frac{1}{6} $ & $ E_{1} = 1 $ \\
	$ B_{2} = \frac{1}{30} $ & $ E_{2} = 5 $ \\
	$ B_{3} = \frac{1}{42} $ & $ E_{3} = 61 $ \\
	$ B_{4} = \frac{1}{30} $ & $ E_{4} = 1385 $ \\
	$ B_{5} = \frac{5}{66}  $ & $ E_{5} = 50521 $ \\
	$ B_{6} = \frac{691}{2730} $ & $ E_{6} = 2702765 $ \\
	$ B_{7} = \frac{7}{6} $ & $ E_{7} = 199360981 $ \\
	$ B_{8} = \frac{3617}{510} $ & $ E_{8} = 19391 512 145 $ \\
	$ B_{9} = \frac{43867}{798} $ & $ E_{9} = 2 404 879 675 441 $ \\
	$ B_{10} = \frac{174611}{330} $ & $ E_{10} = 370 371 188 237 525 $ \\
	\bottomrule
\end{tabular}


\section{Σχέσεις μεταξύ των αριθμών \textlatin{Bernoulli} και \textlatin{Euler}}

\begin{tabular}{l}
$ \binom{2n + 1}{2} 2^{2} B_{1} - \binom{2n+1}{4} 2^{4} B_{2} +
	\binom{2n+1}{6} 2^{6} B_{3} - \cdots {(-1)}^{n-1}{(2n+1)}^{2n} B_{n} = 2n $ \\
	$ E_{n} = \binom{2n}{2} E_{n-1} - \binom{2n}{4} E_{n-2} +
	\binom{2n}{6} E_{n-3} - \cdots {(-1)}^{n} $ \\
	$ B_{n} = \frac{2n}{2^{2n} (2^{2n}-1)} \left\{ \binom{2n-1}{1} E_{n-1}
	- \binom{2n-1}{3} E_{n-2} + \binom{2n-1}{5} E_{n-3} - \cdots {(-1)}^{n-1}\right\} $ 
\end{tabular}

\begin{tabular}{l}
	$ B_{n} = \frac{(2n)!}{2^{2n-1}\pi^{2n}}\left\{ 1 + \frac{1}{2^{2n}} +
\frac{1}{3^{2n}} + \cdots\right\}  $ \\
$ B_{n} = \frac{2(2n)!}{(2^{2n} \pi ^{2n})} \left\{ 1 + \frac{1}{3^{2n}} +
	\frac{1}{5^{2n}} + \cdots\right\} $ \\
	$ B_{n} = \frac{2(2n)!}{(2^{2n-1}-1)\pi^{2n}} \left\{ 1 - \frac{1}{2^{2n}}
	+ \frac{1}{3^{2n}} - \cdots\right\} $ \\ 
	$ E_{n} = \frac{2^{2n+2}(2n)!}{\pi ^{2n+1}} \left\{ 1 - \frac{1}{3^{2n+1}} +
	\frac{1}{5^{2n+1}} - \cdots\right\} $
\end{tabular}

\section{Ασυμπτωτική προσέγγιση για τους αριθμούς \textlatin{Bernoulli}}

\[
	B_{n} \sim 4n^{2n}{(\pi e)}^{-2n}\sqrt{\pi n} 
\]
\end{document}

