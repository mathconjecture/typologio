
\section{Ορισμός των αριθμών \textlatin{Bernoulli}}

Οι αριθμοί \textlatin{Bernoulli} $ B_{1} $, $ B_{2} $, $ B_{3}, \ldots $   
ορίζονται από τις σειρές

\begin{tabular}{ll}
	$ \frac{x}{e^{x} - 1} = 1 - \frac{x}{2} + \frac{B_{1} x^{2}}{2!} -
	\frac{B_{2} x^{4}}{4!} + \frac{B_{3} x^{6}}{6!} - \cdots $ & $ \abs{x} < 2
	\pi $ \\
	$ 1 - \frac{x}{2} \cot{\frac{x}{2}} = \frac{B_{1} x^{2}}{2!} +
	\frac{B_{2} x^{4}}{4!} + \frac{B_{3} x^{6}}{6!} + \cdots $ & $ \abs{x} < \pi $
\end{tabular}

\section{Ορισμός των αριθμών \textlatin{Euler}}

Οι αριθμοί \textlatin{Euler} $ E_{1} $, $ E_{2} $, $ E_{3}, \ldots $ ορίζονται
από τις σειρές

\begin{tabular}{ll}
	$ \sech{x} = 1 - \frac{E_{1} x^{2}}{2!} + \frac{E_{2} x^{4}}{4!} -
	\frac{E_{3} x^{6}}{6!} + \cdots $ & $ \abs{x} < \frac{\pi}{2} $ \\
	$ \sec{x} = 1 + \frac{E_{1} x^{2}}{2!} + \frac{E_{2} x^{4}}{4!} -
	\frac{E_{3} x^{6}}{6!} + \cdots $ & $ \abs{x} < \frac{\pi}{2}  $
\end{tabular}


\section{Πίνακας μερικών αριθμών \textlatin{Bernoulli} και \textlatin{Euler}}

\begin{tabular}{ll}
	\toprule \\
	\textlatin{Bernoulli} & \textlatin{Euler} \\
	\midrule \\
	$ B_{1} = \frac{1}{6} $ & $ E_{1} = 1 $ \\
	$ B_{2} = \frac{1}{30} $ & $ E_{2} = 5 $ \\
	$ B_{3} = \frac{1}{42} $ & $ E_{3} = 61 $ \\
	$ B_{4} = \frac{1}{30} $ & $ E_{4} = 1385 $ \\
	$ B_{5} = \frac{5}{66}  $ & $ E_{5} = 50521 $ \\
	$ B_{6} = \frac{691}{2730} $ & $ E_{6} = 2702765 $ \\
	$ B_{7} = \frac{7}{6} $ & $ E_{7} = 199360981 $ \\
	$ B_{8} = \frac{3617}{510} $ & $ E_{8} = 19391 512 145 $ \\
	$ B_{9} = \frac{43867}{798} $ & $ E_{9} = 2 404 879 675 441 $ \\
	$ B_{10} = \frac{174611}{330} $ & $ E_{10} = 370 371 188 237 525 $ \\
	\bottomrule
\end{tabular}


\section{Σχέσεις μεταξύ των αριθμών \textlatin{Bernoulli} και \textlatin{Euler}}

\begin{tabular}{l}
$ \binom{2n + 1}{2} 2^{2} B_{1} - \binom{2n+1}{4} 2^{4} B_{2} +
	\binom{2n+1}{6} 2^{6} B_{3} - \cdots {(-1)}^{n-1}{(2n+1)}^{2n} B_{n} = 2n $ \\
	$ E_{n} = \binom{2n}{2} E_{n-1} - \binom{2n}{4} E_{n-2} +
	\binom{2n}{6} E_{n-3} - \cdots {(-1)}^{n} $ \\
	$ B_{n} = \frac{2n}{2^{2n} (2^{2n}-1)} \left\{ \binom{2n-1}{1} E_{n-1}
	- \binom{2n-1}{3} E_{n-2} + \binom{2n-1}{5} E_{n-3} - \cdots {(-1)}^{n-1}\right\} $ 
\end{tabular}

\begin{tabular}{l}
	$ B_{n} = \frac{(2n)!}{2^{2n-1}\pi^{2n}}\left\{ 1 + \frac{1}{2^{2n}} +
\frac{1}{3^{2n}} + \cdots\right\}  $ \\
$ B_{n} = \frac{2(2n)!}{(2^{2n} \pi ^{2n})} \left\{ 1 + \frac{1}{3^{2n}} +
	\frac{1}{5^{2n}} + \cdots\right\} $ \\
	$ B_{n} = \frac{2(2n)!}{(2^{2n-1}-1)\pi^{2n}} \left\{ 1 - \frac{1}{2^{2n}}
	+ \frac{1}{3^{2n}} - \cdots\right\} $ \\ 
	$ E_{n} = \frac{2^{2n+2}(2n)!}{\pi ^{2n+1}} \left\{ 1 - \frac{1}{3^{2n+1}} +
	\frac{1}{5^{2n+1}} - \cdots\right\} $
\end{tabular}

\section{Ασυμπτωτική προσέγγιση για τους αριθμούς \textlatin{Bernoulli}}

\[
	B_{n} \sim 4n^{2n}{(\pi e)}^{-2n}\sqrt{\pi n} 
\]
