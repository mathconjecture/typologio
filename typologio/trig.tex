\documentclass[a4paper,10pt]{report}
\usepackage{etex}
%%%%%%%%%%%%%%%%%%%%%%%%%%%%%%%%%%%%%%
% Babel language package
\usepackage[english,greek]{babel}
% Inputenc font encoding
\usepackage[utf8]{inputenc}
%%%%%%%%%%%%%%%%%%%%%%%%%%%%%%%%%%%%%%

\usepackage{extsizes}
\usepackage{multicol}

%%%%% math packages %%%%%%%%%%%%%%%%%%
\usepackage[intlimits]{amsmath}
\usepackage{amssymb}
\usepackage{amsfonts}
\usepackage{amsthm}
\usepackage{proof}
\usepackage{mathtools}

\usepackage[italicdiff]{physics}
\usepackage{siunitx}
\usepackage{xfrac}

%%%%%%% symbols packages %%%%%%%%%%%%%%
\usepackage{bm} %for use \bm instead \boldsymbol in math mode
\usepackage{dsfont}
\usepackage{stmaryrd}
%%%%%%%%%%%%%%%%%%%%%%%%%%%%%%%%%%%%%%%


%%%%%% graphics %%%%%%%%%%%%%%%%%%%%%%%
\usepackage{graphicx}
\usepackage{color}
%\usepackage{xypic}
%\usepackage[all]{xy}
%\usepackage{calc}

%%%%%% tables %%%%%%%%%%%%%%%%%%%%%%%%%
\usepackage{array}
\usepackage{booktabs}
\usepackage{multirow}
\usepackage{makecell}
\usepackage{minibox}
\usepackage{systeme}
%%%%%%%%%%%%%%%%%%%%%%%%%%%%%%%%%%%%%%%

\usepackage{enumerate}

\usepackage{fancyhdr}
%%%%% header and footer rule %%%%%%%%%
\setlength{\headheight}{14pt}
\renewcommand{\headrulewidth}{0pt}
\renewcommand{\footrulewidth}{0pt}
\fancypagestyle{plain}{\fancyhf{}
\fancyhead{}
\lfoot{}
\rfoot{\small \thepage}}
\fancypagestyle{vangelis}{\fancyhf{}
\rhead{\small \leftmark}
\lhead{\small }
\lfoot{}
\rfoot{\small \thepage}}
%%%%%%%%%%%%%%%%%%%%%%%%%%%%%%%%%%%%%%%

%\usepackage{hyperref}
%\usepackage{url}
%%%%%%%% hyperref settings %%%%%%%%%%%%
%\hypersetup{pdfpagemode=UseOutlines,hidelinks,
%bookmarksopen=true,
%pdfdisplaydoctitle=true,
%pdfstartview=Fit,
%unicode=true,
%pdfpagelayout=OneColumn,
%}
%%%%%%%%%%%%%%%%%%%%%%%%%%%%%%%%%%%%%%%

\usepackage[space]{grffile}

\usepackage{geometry}
\geometry{left=25.63mm,right=25.63mm,top=36.25mm,bottom=36.25mm,footskip=24.16mm,headsep=24.16mm}

%\usepackage[explicit]{titlesec}
%%%%%% titlesec settings %%%%%%%%%%%%%
%\titleformat{\chapter}[block]{\LARGE\sc\bfseries}{\thechapter.}{1ex}{#1}
%\titlespacing*{\chapter}{0cm}{0cm}{36pt}[0ex]
%\titleformat{\section}[block]{\Large\bfseries}{\thesection.}{1ex}{#1}
%\titlespacing*{\section}{0cm}{34.56pt}{17.28pt}[0ex]
%\titleformat{\subsection}[block]{\large\bfseries{\thesubsection.}{1ex}{#1}
%\titlespacing*{\subsection}{0pt}{28.80pt}{14.40pt}[0ex]
%%%%%%%%%%%%%%%%%%%%%%%%%%%%%%%%%%%%%%

%%%%%%%%% My Theorems %%%%%%%%%%%%%%%%%%
\newtheorem{thm}{Θεώρημα}[section]
\newtheorem{cor}[thm]{Πόρισμα}
\newtheorem{lem}[thm]{λήμμα}
\theoremstyle{definition}
\newtheorem{dfn}{Ορισμός}[section]
\newtheorem{dfns}[dfn]{Ορισμοί}
\newtheorem{ex}[thm]{Παραδειγμα}
\theoremstyle{remark}
\newtheorem{remark}{Παρατήρηση}[section]
\newtheorem{remarks}[remark]{Παρατηρήσεις}
%%%%%%%%%%%%%%%%%%%%%%%%%%%%%%%%%%%%%%%

%%%%%%% nesting newcommands $$$$$$$$$$$$$$$$$$$
\newcommand{\function}[1]{\newcommand{\nvec}[2]{#1(##1_1,\ldots, ##1_##2)}}

\newcommand{\linode}[2]{#1_n(x)#2^{(n)}+#1_{n-1}(x)#2^{(n-1)}+\cdots +#1_0(x)#2=g(x)} 
\newcommand{\vecoffun}[3]{#1_0(#2),\ldots,#1_#3(#2)}

\newcommand{\mysum}[1]{\sum_{n=#1}^{\infty}}



\renewcommand{\vector}[1]{(x_1,x_2,\ldots,x_{#1})}
\newcommand{\avector}[2]{(#1_1,#1_2,\ldots,#1_{#2})}
\newcommand{\aDEFvector}[2][a]{(#1_1,#1_2,\ldots,#1_{#2})}

\newcommand{\rt}[3]{\vb{r}(t)=#1\,\vb{i}+#2\,\vb{j}+#3\,\vb{k}}
\newcommand{\rtt}[2]{\vb{r}(t)=#1\,\vb{i}+#2\,\vb{j}}
\newcommand{\rs}[3]{\vb{r}(s)=#1\,\vb{i}+#2\,\vb{j}+#3\,\vb{k}}
\newcommand{\rss}[2]{\vb{r}(s)=#1\,\vb{i}+#2\,\vb{j}}
\newcommand{\vect}[4]{\vb{#1}=#2\,\vb{i}+#3\,\vb{j}+#4\,\vb{k}}
\newcommand{\vectt}[3]{\vb{#1}=#2\,\vb{i}+#3\,\vb{j}}


\DeclareMathOperator{\Arg}{Arg}

%%%%%%%%% My Theorems %%%%%%%%%%%%%%%%%%
\newtheorem{thm}{Θεώρημα}[section]
\newtheorem{cor}[thm]{Πόρισμα}
\newtheorem{lem}[thm]{λήμμα}
\theoremstyle{definition}
\newtheorem{dfn}{Ορισμός}[section]
\newtheorem{dfns}[dfn]{Ορισμοί}
\newtheorem{ex}[thm]{Παραδειγμα}
\theoremstyle{remark}
\newtheorem{remark}{Παρατήρηση}[section]
\newtheorem{remarks}[remark]{Παρατηρήσεις}
%%%%%%%%%%%%%%%%%%%%%%%%%%%%%%%%%%%%%%%



    \begin{document}




\section{Ορισμός Τριγωνομετρικών αριθμών οξείας γωνίας}

\begin{tabular}{l>{$}l<{$}}
    Ημίτονο & \sin A = \frac{a}{c} = \frac{\text{απέναντι}}{\text{υποτείνουσα}} \\
    Συνημίτονο & \cos A = \frac{b}{c} =
    \frac{\text{προσκείμενη}}{\text{υποτείνουσα}} \\
    Εφαπτομένη & \tan A = \frac{a}{b} =
    \frac{\text{απέναντι}}{\text{προσκείμενη}}\\
    Συνεφαπτομένη & \cot A = \frac{b}{a} =
    \frac{\text{προσκείμενη}}{\text{απέναντι}} \\
    Τέμνουσα & \sec A = \frac{c}{b} =
    \frac{\text{υποτείνουσα}}{\text{προσκείμενη}}\\
    Συντέμνουσα & \csc a = \frac{c}{a} =
    \frac{\text{υποτείνουσα}}{\text{απέναντι}} \\
\end{tabular}  

\section{Ορισμός Τριγωνομετρικών συναρτήσεων}

\begin{tabular}{l}
    $ \sin A = \frac{y}{r} $ \\
    $ \cos A = \frac{x}{r} $ \\
    $ \tan A = \frac{y}{x} $ \\
    $ \cot A = \frac{x}{y} $ \\
    $ \sec A = \frac{r}{x} $ \\
    $ \csc A = \frac{r}{y} $ 
\end{tabular}

\section{Σχέση μεταξύ μοιρών και ακτινίων}

Η γωνία $ \theta $ σε ακτίνια ορίζεται ως 
\[
    \theta = \frac{\text{μήκος τόξου AB}}{\text{ακτίνα}} = \frac{l}{r}
\] 

Η σχέση η οποία συνδέει μοίρες και ακτίνια είναι 
\[ 
    \frac{\SI{}[\mu]{\degree}}{\theta} = \frac{\pi}{\ang{180}} 
\]


\begin{itemize}
    \item $\SI{1}{radian} = \frac{\ang{180}}{\pi} =
        \SI{57.29577951308232\dots}{\degree} $    
    \item \SI{1}{\degree} = \SI{\pi/180}{radians} =
        \SI{0.0174532925199432957692\dots}{radians}
\end{itemize}

\section{Σχέσεις μεταξύ Τριγωνομετρικών συναρτήσεων}


\begin{tabular}{ll}
    $ \tan{A} = \frac{\sin{A}}{\cos{A}} $ & $ \sin^{2}{A} + \cos^{2}{A} =1 $ \\
    $ \cot{A} = \frac{1}{\tan{A}} = \frac{\cos{A}}{\sin{A}} $ & $
    \sec^{2}{A} - \tan^{2}{A} = 1 $ \\
    $ \sec{A} = \frac{1}{\cos{A}} $ & $ \csc^{2}{A} - \cot^{2}{A} =1 $ \\
    $ \csc{A} = \frac{1}{\sin{A}} $
\end{tabular}


\begin{table}
    \centering
\begin{tabular}{cccccccc}
    \toprule \\
    \minibox[c]{Γωνία Α \\ (μοίρες)} &  \minibox{Γωνία Α \\ (ακτίνια)} & $ \sin{A} $ & $ \cos{A} $ & $ \tan{A} $ &  $ \cot{A} $ & $ \sec{A} $ & $ \csc{A} $ \\
    \midrule \\
    \ang{0} & 0 & 0 & 1 & 0 & $\infty$ & 1 & $\infty$ \\
    \ang{15} & $ \pi / 12 $ & $ \frac{\sqrt{6} - \sqrt{2}}{4} $ & $ \frac{\sqrt{6} + \sqrt{2}}{4} $ & $ 2 - \sqrt{3} $ & $ 2 + \sqrt{3} $ & $ \sqrt{6} - \sqrt{2} $ & $ \sqrt{6} + \sqrt{2} $ \\
    \ang{30} & $ \pi / 6 $ & $ \frac{1}{2} $ & $ \frac{\sqrt{3}}{2} $ & $ \frac{\sqrt{3}}{3} $ & $ \sqrt{3} $ & $ \frac{2 \sqrt{3}}{3} $ & $2$ \\
    \ang{45} & $ \pi / 4 $ & $ \frac{\sqrt{2}}{2} $ & $ \frac{\sqrt{2}}{2} $ & $1$ & $1$ & $ \sqrt{2} $ & $ \sqrt{2} $ \\
    \ang{60} & $ \pi / 3 $ & $ \frac{\sqrt{3}}{2} $ & $ \frac{1}{2} $ & $ \sqrt{3} $ & $ \frac{\sqrt{3}}{3} $ & $ 2 $ & $ \frac{2 \sqrt{3}}{3} $ \\   
    \ang{75} & $ \frac{5 \pi}{12} $ & $ \frac{\sqrt{6} + \sqrt{2}}{4} $ & $ \frac{\sqrt{6} - \sqrt{2}}{4} $ & $ 2 + \sqrt{3} $ & $ 2 - \sqrt{3} $ & $ \sqrt{6} + \sqrt{2} $ &  $ \sqrt{6} - \sqrt{2} $ \\
    \ang{90} & $ \frac{\pi}{2} $ & $ 1 $ & $ 0 $ & $ \pm \infty $ & $ 0 $ & $ \pm \infty $ & $ 1 $ \\
    \ang{105} & $ \frac{7 \pi}{12} $ & $ \frac{\sqrt{6} + \sqrt{2}}{4} $ & $ - \frac{\sqrt{6} - \sqrt{2}}{4} $ &  $ -( 2 + \sqrt{3} ) $ & $ -2( 2 - \sqrt{3}) $ & $ -(\sqrt{6} + \sqrt{2}) $ & $ \sqrt{6} -
    \sqrt{2} $ \\
    \ang{120} & $ \frac{2 \pi}{3} $ & $ \frac{\sqrt{3}}{2} $ & $ - \frac{1}{2} $ & $ - \sqrt{3} $ & $ - \frac{\sqrt{3}}{3} $ & $ -2 $ & $ \frac{2 \sqrt{3}}{3} $ \\
    \ang{135} & $ \frac{3 \pi}{4} $ & $ \frac{\sqrt{2}}{2} $ & $ - \frac{\sqrt{2}}{2} $ & $ -1 $ & $ -1 $ & $ - \sqrt{2} $ & $ \sqrt{2} $ \\
    \ang{150} & $ \frac{5 \pi}{6} $ & $ \frac{1}{2} $ & $ - \frac{\sqrt{3}}{2} $ & $ - \frac{\sqrt{3}}{3} $ & $ - \sqrt{3} $ & $ - \frac{2 \sqrt{3}}{3} $ & $ 2 $ \\ 
    \ang{165} & $ \frac{11 \pi}{12} $ & $ \frac{\sqrt{6} - \sqrt{2}}{4} $ & $ - \frac{\sqrt{6} + \sqrt{2}}{4} $ & $ - ( 2 - \sqrt{3} ) $ & $ -2 ( 2 + \sqrt{3} ) $ & $ -(\sqrt{6} - \sqrt{2}) $ & $ \sqrt{6}
    + \sqrt{2} $ \\ 
    \ang{180} & $ \pi $ & $ 0 $ & $ -1 $ & $ 0 $ & $ \mp \infty $ & $ -1 $ & $ \pm \infty $ \\
    \ang{195} & $ \frac{13 \pi}{12} $ & $ - \frac{\sqrt{6} - \sqrt{2}}{4} $ & $
    - \frac{\sqrt{6} + \sqrt{2}}{4} $ & $  2 - \sqrt{3} $ & $  2 + \sqrt{3} $ &  $ - (\sqrt{6} - \sqrt{2}) $ & $ - (\sqrt{6} + \sqrt{2}) $ \\ 
    \ang{210} & $ \frac{7 \pi}{6} $ & $ - \frac{1}{2} $ & $ - \frac{\sqrt{3}}{2} $ & $ \frac{\sqrt{3}}{3} $ & $ \sqrt{3} $ & $ - \frac{2 \sqrt{3}}{3} $ & $ -2 $ \\
    \ang{225} & $ \frac{5 \pi}{4} $ & $ - \frac{\sqrt{2}}{2} $ & $ - \frac{\sqrt{2}}{2} $ & $ 1 $ & $ 1 $ & $ - \sqrt{2} $ & $ - \sqrt{2} $ \\
    \ang{240} & $ \frac{4 \pi}{3} $ & $ - \frac{\sqrt{3}}{2} $ & $ - \frac{1}{2} $ & $ \sqrt{3} $ & $ \frac{\sqrt{3}}{3} $ & $ -2 $ & $ - \frac{2 \sqrt{3}}{3} $ \\
    \ang{255} & $ \frac{17 \pi}{12} $ & $ - \frac{\sqrt{6} + \sqrt{2}}{4} $ & $ - \frac{\sqrt{6} - \sqrt{2}}{4} $ & $  2 + \sqrt{3} $ & $  2 - \sqrt{3} $ & $ - (\sqrt{6} + \sqrt{2}) $ & $ - (\sqrt{6} -
    \sqrt{2}) $ \\
    \ang{270} & $ \frac{3 \pi}{2} $ & $ -1 $ & $ 0 $ & $ \pm \infty $ & $ 0 $ & $ \mp \infty $ & $ -1 $ \\
    \ang{285} & $ \frac{19 \pi}{12} $ & $ - \frac{\sqrt{6} + \sqrt{2}}{4} $ & $ \frac{\sqrt{6} - \sqrt{2}}{4} $ & $ - ( 2 + \sqrt{3} ) $ & $ - ( 2 - \sqrt{3}) $ & $ \sqrt{6} + \sqrt{2} $ & $ -(\sqrt{6} -
    \sqrt{2}) $ \\
    \ang{300} & $ \frac{5 \pi}{3} $ & $ - \frac{\sqrt{3}}{2} $ & $ \frac{1}{2} $ & - $\sqrt{3} $ & $ - \frac{\sqrt{3}}{3} $ & $ 2 $ & $ - \frac{2 \sqrt{3}}{3} $ \\    
    \ang{315} & $ \frac{7 \pi}{4} $ & $ - \frac{\sqrt{2}}{2} $ & $ \frac{\sqrt{2}}{2} $ & $ -1 $ & $ -1 $ $ \sqrt{2} $ & $ - \sqrt{2} $ \\
    \ang{330} & $ \frac{11 \pi}{6} $ & $ - \frac{1}{2} $ & $ \frac{\sqrt{3}}{2}$ & $ - \frac{\sqrt{3}}{3} $ & $ - \sqrt{3} $ & $ \frac{2\sqrt{3}}{3} $ & $ - 2 $ \\ 
    \ang{345} & $ \frac{23 \pi}{12} $ & $ - \frac{\sqrt{6} - \sqrt{2}}{4} $ & $ \frac{\sqrt{6} + \sqrt{3}}{4} $ & $ - ( 2 - \sqrt{3}) $ & $ -2 ( 2 + \sqrt{3}) $ & $ \sqrt{6} - \sqrt{2} $ & $ 
    - (\sqrt{6} + \sqrt{2}) $ \\
    \ang{360} & $ 2 \pi $ & $ 0 $ & $ 1 $ & $ 0 $ & $ \mp \infty $ & $ 1 $ & $ \mp \infty $ \\
    \bottomrule
\end{tabular}
\end{table}

\section{Γραφικές Παραστάσεις των τριγωγωνομετρικών συναρτήσεων}


\section{Συναρτήσεις Αρνητικών γωνιών}
\begin{tabular}{l} 
    $ \sin{(-A)} = - \sin{A} $ \\
    $ \cos{(-A)} = \cos{A} $ \\
    $ \tan{(-A)} = - \tan{A} $ \\
    $ \cot{(-A)} = - \cot{A} $ \\
    $ \sec{(-A)} = \sec{A} $ \\
    $ \csc{(-A)} = - \csc{A} $ 
\end{tabular}


\section{Τριγωνομετρικοί αριθμοί Αθροίσματος} 

\begin{tabular}{l}
    $ \sin{(A+B)} = \sin{A} \cos{B} + \cos{A} \sin{B} $ \\
    $ \sin{(A-B)} = \sin{A} \cos{B} - \cos{A} \sin{B} $ \\
    $ \cos{(A+B)} = \cos{A} \cos{B} - \sin{A} \sin{B} $ \\
    $ \cos{(A-B)} = \cos{A} \cos{B} + \sin{A} \sin{B} $ \\
    $ \tan{(A+B)} = \frac{\tan{A} + \tan{B}}{1 - \tan{A} \tan{B}} $ \\
    $ \tan{(A-B)} = \frac{\tan{A} - \tan{B}}{1 + \tan{A} \tan{B}} $ \\ 
    $ \cot{(A+B)} = \frac{\cot{A} \cot{B} - 1}{\cot{B} + \cot{A}} $ \\
    $ \cot{(A-B)} = \frac{\cot{A} \cot{B} + 1}{\cot{B} - \cot{A}} $ 
\end{tabular}

\section{Αναγωγή στο Πρώτο Τεταρτημόριο}

\begin{center}
\begin{tabular}{*{6}c}
    \toprule \\
    & $ -A $  & $ \frac{\pi}{2} \pm A $ & $ \pi \pm A $ & $ \frac{3 \pi}{2}
    \pm A $ & $ 2k \pi \pm A $ \\
    \midrule \\
    $ \sin{} $ & $ - \sin{A} $ & $ \cos{A} $ & $ \sin{A} $ & $ - \cos{A}
    $ & $ \pm \sin{A} $a \\
    $ \cos $ & $ \cos{A}  $ & $ \mp \sin{A} $ & $ - \cos{A} $ & $ \mp
    \sin{A} $ & $ \cos{A} $ \\
    $ \tan{} $ & $ - \tan{A} $ & $ \mp \cot{A} $ & $ \pm \tan{A} $ & $ \mp
    \cot{A} $ & $ \pm \tan{A} $ \\
    $ \sec{A} $ & $ \sec{A} $ & $ \mp \csc{A} $ & $ - \sec{A} $ & $ \pm
    \csc{} $ & $ \sec{A} $ \\
    $ \csc{} $ & $ - \csc{A} $ & $ \sec{A} $ & $ \mp \csc{A} $ & $ - \sec{A}$ &
    $ \pm \csc{A} $ \\
    $ \cot{} $ & $ - \cot{A} $ & $ \mp \tan{A} $ & $ \pm \cot{A} $ & $ \mp
    \tan{A} $ & $ \pm \cot{A} $ \\ 
    \bottomrule
\end{tabular}
\end{center}


\section{Σχέσεις μεταξύ τριγωνομετρικών συναρτήσεων}

\begin{center}
    \begin{tabular}{*{7}c}
        \toprule \\
    & $ \sin{A} = u $ & $ \cos{A} = u $ & $ \tan{A} = u $ & $ \sec{A} = u $ & $
    \csc{A} = u $ \\
    \midrule \\
        $ \sin{A} $ & $ u $ & $ \sqrt{1-u^{2}} $ & $ \frac{u}{\sqrt{1 + u^{2}}
    }$ & $ \frac{\sqrt{u^{2}-1} }{u} $ & $ \frac{1}{u} $ \\ 
        $ \cos{A} $ & $ \sqrt{1-u^{2}} $ & $ u $ & $ \frac{1}{\sqrt{1 + u^{2}}}
        $ & $ \frac{u}{\sqrt{1 + u^{2}}} $ & $ \frac{1}{u} $ & $
        \frac{\sqrt{u^{2}-1}}{u} $ \\  
        $ \tan{A} $ & $ \frac{u}{ \sqrt{1-u^{2}}} $ & $ \frac{ \sqrt{1-u^{2}}}{
        u} $ & $ u $ & $ \frac{1}{u} $ & $ \sqrt{u^{2}-1} $ & $
        \frac{1}{\sqrt{u^{2}-1}} $ \\
        $ \cot{A} $ & $ \frac{ \sqrt{1-u^{2}}}{u} $ & $ \frac{u}{
    \sqrt{1-u^{2}}} $ & $ \frac{1}{u} $ & $ u $ & $ \frac{1}{\sqrt{u^{2}-1}} $ & $
    \sqrt{u^{2}-1} $ \\
        $ \sec{A} $ & $ \frac{1}{ \sqrt{1-u^{2}}} $ & $ \frac{1}{u} $ & 
        $\sqrt{1 + u^{2}} $ & $ \frac{\sqrt{1 + u^{2}}}{u} $ & $ u $ & 
            $ \frac{u}{\sqrt{u^{2}-1}} $ \\
        $ \csc{A} $ & $ \frac{1}{u} $ & $ \frac{1}{\sqrt{1-u^{2}}} $ &
        $ \frac{\sqrt{1 + u^{2}}}{u} $ & $\sqrt{1 + u^{2}}$ &
        $ \frac{u}{\sqrt{u^{2}-1}} $ & $ u $ \\
        \bottomrule
    \end{tabular}
\end{center}


\section {Γωνίες Διπλασίου Τόξου}

\begin{tabular}{l}
    $ \sin{2A} = 2 \sin{A} \cos{A} $ \\
    $ \cos{2A} = \cos^{2}{A} - \sin^{2}{A} = 1 - 2 \sin^{2}{A} = 2
    \cos^{2}{A} -1 $ \\
    $ \tan{2A} = \frac{2 \tan{A}}{1 - \tan^{2}{A}} $ 
\end{tabular}

\section{Γωνίες Υποδιπλάσιου Τόξου}


\begin{tabular}{>{$}l<{$}}
     \sin{\frac{A}{2}} = \begin{cases}
        \sqrt{\frac{1 - \cos{A}}{2}} \qq{αν $\frac{A}{2}$ είναι στο 1ο ή 2ο 
        Τεταρτημόριο} \\
        -\sqrt{\frac{1 - \cos{A}}{2}} \qq{αν $\frac{A}{2}$ είναι στο 3ο ή 4ο
        Τεταρτημόριο}
\end{cases} \\
    \cos{\frac{A}{2}} = \begin{cases}
        \sqrt{\frac{1 + \cos{A}}{2}} \qq{αν $\frac{A}{2}$ είναι στο 1ο ή 4ο 
        Τεταρτημόριο} \\
        -\sqrt{\frac{1 + \cos{A}}{2}} \qq{αν $\frac{A}{2}$ είναι στο 2ο ή 3ο
        Τεταρτημόριο}
    \end{cases} \\
    \begin{aligned}
         \tan{\frac{A}{2}} &= \begin{cases}
      \sqrt{\frac{1 - \cos{A}}{1 + \cos{A}}} \qq{αν $\frac{A}{2}$ είναι στο
            1ο ή 3ο Τεταρτημόριο} \\
        -\sqrt{\frac{1 - \cos{A}}{1 + \cos{A}}} \qq{αν $\frac{A}{2}$ είναι στο
            2ο ή 4ο Τεταρτημόριο} 
    \end{cases} \\
                            &= \frac{\sin{A}}{1 + \cos{A}} = \frac{1 -
                            \cos{A}}{\sin{A}} = \csc{A} - \cot{A} 
    \end{aligned}
\end{tabular}


\section{Γωνίες Πολλαπλάσιου Τόξου}

\begin{tabular}{l}
$ \sin{3A} = 3 \sin{A} - 4 \sin^{3}{A} $ \\
$ \cos{3A} = 4 \cos^{3}{A} - 3 \cos{A} $ \\
$ \tan{3A} = \frac{3 \tan{A} - \tan^{3}{A}}{1 - 3 \tan^{2}{A}} $ \\
$ \sin{4A} = 4 \sin{A} \cos{A} - 8 \sin^{3}{A} \cos{A} $ \\
$ \cos{4A} = 8 \cos^{4}A - 8 \cos^{2}{A} + 1 $ \\
$ \tan{4A} = \frac{4 \tan{A} - 4 \tan^{3}{A}}{1 - 6 \tan^{2}{A} + \tan^{4}A} $
\\
\end{tabular}

\section{Δυνάμεις Τριγωνομετρικών Συναρτήσεων}

\begin{tabular}{l}
    $ \sin^{2}{A} = \frac{1 - \cos{2A}}{2} $ \\
    $ \cos^{2}{A} = \frac{1 + \cos{2A}}{2} $ \\
    $ \sin^{3}{A} = \frac{3 \sin{A} - \sin{3A}}{4} $ \\
    $ \cos^{3}{A} = \frac{3 \cos{A} + \cos{3A}}{4} $ \\
    $ \sin^{4}{A} = \frac{3 - 4 \cos{2A} + \cos{4A}}{8} $ \\
    $ \cos^{4}{A} = \frac{3 + 4 \cos{2A} + \cos{4A}}{8} $ 
\end{tabular}


\section{Aθροίσματα Διαφορές και Γινόμενα Τριγωνομετρικών Συναρτήσεων}  
   \begin{tabular}{l}
       $ \sin{A} + \sin{B} =2 \sin{\frac{A + B}{2} \cos{\frac{A + B}{2}}}  $ \\ 
       $ \sin{A} - \sin{B} = 2 \cos{\frac{A + B}{2}} \sin{\frac{A - B}{2}}  $ \\        
       $ \cos{A} + \cos{B} = 2 \cos{\frac{A + B}{2}} \cos{\frac{A - B}{2}}  $ \\
       $ \cos{A} - \cos{B} = 2 \sin{\frac{A + B}{2} \sin{\frac{B - A}{2}}} $ \\
       $ \sin{A} \sin{B} = \frac{1}{2} \{\cos{(A-B)} - \cos{(A-B)}\} $ \\
       $ \cos{A} \cos{B} = \frac{1}{2} \{\cos{(A-B)} + \cos{(A+B)}\} $ \\
       $ \sin{A} \cos{B} = \frac{1}{2} \{\sin{(A-B)} + \sin{(A+B)}\} $ 
   \end{tabular} 

   \section{Αντίστροφες Τριγωνομετρικές Συναρτήσεις}

   \section{Σχέσεις μεταξύ αντίστροφων Τριγωνομετρικών Συναρτήσεων}

   \begin{tabular}{l}
       $ \sin^{-1}{x} + \cos^{-1}{x} = \frac{\pi}{2} $ \\
       $ \tan^{-1}{x} + \cot^{-1}{x} = \frac{\pi}{2} $ \\
       $ \sec^{-1}{x} + \csc^{-1}{x} = \frac{\pi}{2} $ \\
       $ \sec^{-1}{x} = \cos^{-1}{\frac{1}{x}} $ \\
       $ \csc^{-1}{x} = \sin^{-1}{\frac{1}{x}} $ \\
       $ \cot^{-1}{x} = \tan^{-1}{\frac{1}{x}} $ 
   \end{tabular}

   \begin{tabular}{l}
       $ \sin^{-1}{(-x)} = - \sin^{-1}{x} $ \\
       $ \cos^{-1}{(-x)} = \pi - \cos^{-1}{x} $ \\
       $ \tan^{-1}{(-x)} = - \tan^{-1}{x} $ \\
       $ \cot^{-1}{(-x)} = \pi - \cot^{-1}{x} $ \\
       $ \sec^{-1}{(-x)} = \pi - \sec^{-1}{x} $ \\
       $ \csc^{-1}{(-x)} = - \csc^{-1}{x} $ 
   \end{tabular}
   
\section{Γραφικές Παραστάσεις Αντίστροφων Τριγωνομετρικών Συναρτήσεων}

\section{Σχέσεις μεταξύ πλευρών και γωνιών τριγώνου}

\subsection{Νόμος των Ημιτόνων}
\[
 \frac{a}{\sin{A}} = \frac{b}{\sin{B}} = \frac{c}{\sin{C}} 
\] 

\subsection{Νόμος των Συνημιτόνων}

\begin{align*}
    c^{2} &= a^{2} + b^{2} - 2ab \cos{C} \\
    b^{2} &= a^{2} + c^{2} - 2ac \cos{B} \\
    a^{2} &= b^{2} + c^{2} - 2bc \cos{A} 
\end{align*} 

\subsection{Νόμος των Εφαπτομένων}

\begin{align*} 
   \frac{a + b}{a - b} &= \frac{\tan{(\frac{A + B}{2})}}{\tan{(\frac{A - B}{2}}} 
   \end{align*}

   \[
       \sin{A} = \frac{2}{bc} \sqrt{s(s-a)(s-b)(s-c)} 
   \]
   όπου $ s= \frac{1}{2} (a+b+c) $ είναι η ημιπερίμετρος του τριγώνου.

   "TODO παρόμοιες σχέσεις όπου χρειάζεται

   \end{document}
