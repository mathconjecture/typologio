\subsection{Ιδιότητες των Δυνάμεων}

Αν $p,q$ είναι πραγματικοί αριθμοί, $a,b$ θετικοί πραγματικοί αριθμοί και $m,n$ θετικοί ακέραιοι, τότε:

\begin{align*}
  a^{0}&=1 & a^{1}&=a \\
  a^{p}\cdot a^{q}&=a^{p+q} & \frac{a^{p}}{a^{q}}&=a^{p-q}\\
  (a^{p})^{q}&=a^{p\cdot q} &  a^{-p}&=\frac{1}{a^{p}} \\
  (a\cdot b)^{p}&=a^{p}\cdot b^{p} & \left(\frac{a}{b}\right)^{p}&=\frac{a^{p}}{b^{p}} \\
  \sqrt[n]{a^{m}}&=a^{\frac{m}{n}} & \sqrt[n]{a}&=a^{\frac{1}{n}} \\
  \sqrt[n]{a\cdot b}&=\sqrt[n]{a}\cdot \sqrt[n]{b} & \sqrt[n]{\frac{a}{b}}&=\frac{\sqrt[n]{a}}{\sqrt[n]{b}}
\end{align*}

H συνάρτηση $y=a^{x}$, λέγεται \textbf{\color{blue} εκθετική} συνάρτηση.

\subsection{Λογάριθμοι}

Αν $a^{p}=b$ με $a>0$, $a\neq 1$ και $b>0$ τότε με $p=\log_{a}b$ συμβολίζουμε τον \textbf{λογάριθμο} του $b$ με βάση το $a$.

Η συνάρτηση $y=\log_{a}x$, λέγεται \textbf{\color{blue} λογαριθμική} συνάρτηση και είναι η αντίστροφη της εκθετικής. Δηλαδή ισχύει:
\[
y=a^{x}\Leftrightarrow x=\log_{a}y
\]

\begin{align*}
  \log_{a}1&=0 & \log_{a}a&=1 \\
  \log_{a}(x\cdot y)&=\log_{a}x+\log_{a}y & \log_{a}\left(\frac{x}{y}\right)&=\log_{a}x-\log_{a}y \\
  \log_{a}x^{k}&=k\cdot \log_{a}x & (\log_{a}b)\cdot (\log_{b}a)&=1 \\
\end{align*}

\subsection{Αλλαγή Βάσης}

\[
\log_{a}b=\frac{\log_{c}b}{\log_{c}a}=\log_{a}c\cdot \log_{c}b
\]

Πιο συγκεριμένα, ισχύει:
\begin{gather*}
  \log_{e}b=\ln b = \frac{\log_{10}b}{\log{10}e}=\ln 10\cdot \log_{10}b = {2.30258509294\ldots}\cdot \log_{10}b \\
  \log_{10}b=\log b = \frac{\log_{e}b}{\log_{10}e}=\log_{10}e\cdot \ln b = {0.434294\ldots}\cdot \ln b
\end{gather*}
