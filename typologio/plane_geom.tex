
\section{Απόσταση μεταξύ 2 σημείων}

\[
	d = \sqrt{{(x_{2} - x_{1})}^{2}- {(y_{2} - y_{1})}^{2}}  
\] 


\section{Κλίση $m$ ευθείας που διέρχεται από $ P_{1} (x_{1}, y_{1})$ και $ P_{2}
(x_{2}, y_{2})$}

\[
	m= \frac{y_{2} - y_{1}}{x_{2} - x_{1}} = \tan{\theta}	  
\] 

\section{Εξίσωση ευθείας που διέρχεται από σημεία $ P_{1} (x_{1}, y_{1}) $ και $
P_{2} (x_{2}, y_{2}) $}

\[
	y - y_{1} = m (x - x_{1}) \qq{ή} \frac{y - y_{1}}{x - x_{1}} =
	\frac{y_{2} - y_{1}}{x_{2} - x_{1}} 
\] 

\[
	 y = mx + b 
\]
όπου $ b = y_{1} - m x_{1} = \frac{x_{2} y_{1} - x_{1} y_{2}}{x_{2} - x_{1}}  $
 είναι το σημείο τομής της ευθείας με τον άξονα $ y $.

\section{Εξίσωση ευθείας που τέμνει τους άξονες $x$ και $y$ στα σημεία $ a\neq 0
$ και $ b\neq 0 $ αντίστοιχα}

\[
	\frac{x}{a} + \frac{y}{b} = 1 
\] 

\section{Κανονική μορφή εξίσωσης ευθείας}

\[
	x \cos{a} + y \sin{a} = p 
\] 
όπου $p$ είναι η απόσταση της αρχής των αξόνων $O$ από την ευθεία και $p$ η
γωνία που σχηματίζει η κάθετη απόσταση με τον θετικό ημιάξονα $x$.

\section{Γενική εξίσωση ευθεία}

\[
	 Ax + By + C = 0 
\] 

\section{Απόσταση σημείου $ P(x_{1}, y_{1}) $ από ευθεία $ Ax + By +C = 0 $ }

\[
	\frac{A x_{1} + B y_{1} + C}{\pm \sqrt{A^{2} + B^{2}}} 
\] 
όπου επιλέγεται το κατάλληλο πρόσημο ώστε η απόσταση να είναι μη αρνητική.

\section{Γωνία $ \phi $ μεταξύ ευθειών με κλίσεις $ m_{1} $, $ m_{2} $}

\[
	\tan{\phi} = \frac{m_{2} - m_{1}}{1 + m_{1} m_{2}}  
\] 
Οι ευθείες είναι παράλληλες αν και μόνον αν $ m_{1} = m_{2} $.
Οι ευθείες είναι κάθετες αν και μόνον αν $ m_{1} m_{2} = -1 $.

\section{Εμβαδόν Τριγώνου με κορυφές τα σημεία $ (x_{1}, y_{1}) $, $ (x_{2},
y_{2}) $ και $ (x_{3}, y_{3}) $}

\[
	E = \pm \frac{1}{2} \begin{vmatrix}
		x_{1} & x_{1} & 1 \\
		x_{2} & y_{2} & 1 \\
		x_{3} & y_{3} & 1
	\end{vmatrix} = \pm \frac{1}{2} (x_{1} y_{2} + y_{1} x_{3} - y_{2} x_{3} - y
_{1} x_{2} - x_{1} y_{3})
\] 
όπου επιλέγεται το κατάλληλο πρόσημο ώστε το εμβαδό να είναι μη αρνητικό.
Αν το εμβαδό είναι μηδέν, τότε όλα τα σημεία βρίσκονται πάνω στην ίδια ευθεία.

\section{Μετασχηματισμοί συντεταγμένων}

\subsection{Μεταφορά}

\[
	\begin{cases}
		x = x' + x_{0} \\
		y = y' + y_{0} 
	\end{cases}
	\qq{ή}
	\begin{cases}
		x' = x - x_{0} \\
		y' = y - y_{0} 
	\end{cases}
\] 
όπου $ (x,y) $ και $ (x',y') $ οι συντεταγμένες ως προς τα συστήματα $ xy
$ και $ x'y' $ αντίστοιχα και $ (x_{0}, y_{0}) $ οι συντεταγμένες της αρχής $
O' $ ως προς τη σύστημα $ xy $.

\subsection{Περιστροφή}

\[
	 \begin{cases}
		 x = x' \cos{a} - y' \sin{a} \\
		 y = x' \sin{a} + y' \cos{a} 
	 \end{cases}
	 \qq{ή} 
	 \begin{cases}
		 x' = x \cos{a} + y \sin{a} \\
		 y' = y \cos{a} - x \sin{a}
	 \end{cases}
\]

	 όπου οι αρχές των δύο συστημάτων $ xy $ και $ x'y' $ ταυτίζονται, αλλά ο
	 άξονας $ x' $ σχηματίζει γωνία $ a $ με τον θετικό ημιάξονα $x$.

	 \subsection{Μεταφορά και Περιστροφή}

	 \[
		  \begin{cases}
			  x = x' \cos{a} - y' \sin{a} + x_{0} \\
			  y = x' \sin{a} + y' \cos{a} + y_{0}
		  \end{cases} 
		  \qq{ή}
		  \begin{cases}
			  x' = (x-x_{0}) \cos{a} + (y - y_{0}) \sin{a} \\
			  y' - (y-y_{0}) \cos{a} - (x - x_{0}) \sin{a} 
		  \end{cases}
	 \] 
	 όπου η αρχή $ O' $ του συστήματος $ x'y' $ έχει συντεταγμένες $ (x_{0},
	 y_{0}) $ ως προς το σύστημα $ xy $ και ο άξονας $ x' $ σχηματίζει γωνία $a$
	 με τον θετιό ημιάξονα. 
	



	 \subsection{Πολικές Συντεταγμένες $ (r,\theta) $}

	 Ένα σημείο του επιπέδου μπορεί να προσδιορισθεί είτε με Καρτεσιανές $ (x,y)
	 $ είτε με Πολικές συντεταγμένες $ (r,\theta) $. Οι σχέσεις μετασχηματισμού
	 μεταξύ αυτών των συντεταγμένων είναι


	 \[
		 \begin{cases} 
			 x = r \cos{\theta} \\
			 y = r \sin{\theta}
	 \end{cases}
	 \qq{ή}
	 \begin{cases}
		 r = \sqrt{x^{2} + y^{2}} \\
		 \theta = \tan^{-1}{\frac{y}{x}} 
	 \end{cases}
	 \] 


