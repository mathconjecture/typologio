\documentclass[a4paper,10pt]{report}
\usepackage{etex}
%%%%%%%%%%%%%%%%%%%%%%%%%%%%%%%%%%%%%%
% Babel language package
\usepackage[english,greek]{babel}
% Inputenc font encoding
\usepackage[utf8]{inputenc}
%%%%%%%%%%%%%%%%%%%%%%%%%%%%%%%%%%%%%%

\usepackage{extsizes}
\usepackage{multicol}

%%%%% math packages %%%%%%%%%%%%%%%%%%
\usepackage[intlimits]{amsmath}
\usepackage{amssymb}
\usepackage{amsfonts}
\usepackage{amsthm}
\usepackage{proof}
\usepackage{mathtools}

\usepackage[italicdiff]{physics}
\usepackage{siunitx}
\usepackage{xfrac}

%%%%%%% symbols packages %%%%%%%%%%%%%%
\usepackage{bm} %for use \bm instead \boldsymbol in math mode
\usepackage{dsfont}
\usepackage{stmaryrd}
%%%%%%%%%%%%%%%%%%%%%%%%%%%%%%%%%%%%%%%


%%%%%% graphics %%%%%%%%%%%%%%%%%%%%%%%
\usepackage{graphicx}
\usepackage{color}
%\usepackage{xypic}
%\usepackage[all]{xy}
%\usepackage{calc}

%%%%%% tables %%%%%%%%%%%%%%%%%%%%%%%%%
\usepackage{array}
\usepackage{booktabs}
\usepackage{multirow}
\usepackage{makecell}
\usepackage{minibox}
%%%%%%%%%%%%%%%%%%%%%%%%%%%%%%%%%%%%%%%

\usepackage{enumerate}

\usepackage{fancyhdr}
%%%%% header and footer rule %%%%%%%%%
\setlength{\headheight}{14pt}
\renewcommand{\headrulewidth}{0pt}
\renewcommand{\footrulewidth}{0pt}
\fancypagestyle{plain}{\fancyhf{}
\fancyhead{}
\lfoot{}
\rfoot{\small \thepage}}
\fancypagestyle{vangelis}{\fancyhf{}
\rhead{\small \leftmark}
\lhead{\small }
\lfoot{}
\rfoot{\small \thepage}}
%%%%%%%%%%%%%%%%%%%%%%%%%%%%%%%%%%%%%%%

\usepackage{hyperref}
\usepackage{url}
%%%%%%% hyperref settings %%%%%%%%%%%%
\hypersetup{pdfpagemode=UseOutlines,hidelinks,
bookmarksopen=true,
pdfdisplaydoctitle=true,
pdfstartview=Fit,
unicode=true,
pdfpagelayout=OneColumn,
}
%%%%%%%%%%%%%%%%%%%%%%%%%%%%%%%%%%%%%%

\usepackage[space]{grffile}

\usepackage{geometry}
\geometry{left=25.63mm,right=25.63mm,top=36.25mm,bottom=36.25mm,footskip=24.16mm,headsep=24.16mm}

%\usepackage[explicit]{titlesec}
%%%%%% titlesec settings %%%%%%%%%%%%%
%\titleformat{\chapter}[block]{\LARGE\sc\bfseries}{\thechapter.}{1ex}{#1}
%\titlespacing*{\chapter}{0cm}{0cm}{36pt}[0ex]
%\titleformat{\section}[block]{\Large\bfseries}{\thesection.}{1ex}{#1}
%\titlespacing*{\section}{0cm}{34.56pt}{17.28pt}[0ex]
%\titleformat{\subsection}[block]{\large\bfseries{\thesubsection.}{1ex}{#1}
%\titlespacing*{\subsection}{0pt}{28.80pt}{14.40pt}[0ex]
%%%%%%%%%%%%%%%%%%%%%%%%%%%%%%%%%%%%%%

%%%%%%%%% My Theorems %%%%%%%%%%%%%%%%%%
\newtheorem{thm}{Θεώρημα}[section]
\newtheorem{cor}[thm]{Πόρισμα}
\newtheorem{lem}[thm]{λήμμα}
\theoremstyle{definition}
\newtheorem{dfn}{Ορισμός}[section]
\newtheorem{dfns}[dfn]{Ορισμοί}
\newtheorem{ex}[thm]{Παραδειγμα}
\theoremstyle{remark}
\newtheorem{remark}{Παρατήρηση}[section]
\newtheorem{remarks}[remark]{Παρατηρήσεις}
%%%%%%%%%%%%%%%%%%%%%%%%%%%%%%%%%%%%%%%

%%%%%%% nesting newcommands $$$$$$$$$$$$$$$$$$$
\newcommand{\function}[1]{\newcommand{\nvec}[2]{#1(##1_1,\ldots, ##1_##2)}}

\newcommand{\linode}[2]{#1_n(x)#2^{(n)}+#1_{n-1}(x)#2^{(n-1)}+\cdots +#1_0(x)#2=g(x)}

\newcommand{\vecoffun}[3]{#1_0(#2),\ldots ,#1_#3(#2)}

\newcommand{\mysum}[1]{\sum_{n=#1}^{\infty}}



\renewcommand{\vector}[1]{(x_1,x_2,\ldots,x_{#1})}
\newcommand{\avector}[2]{(#1_1,#1_2,\ldots,#1_{#2})}
\newcommand{\aDEFvector}[2][a]{(#1_1,#1_2,\ldots,#1_{#2})}

\newcommand{\rt}[3]{\vb{r}(t)=#1\,\vb{i}+#2\,\vb{j}+#3\,\vb{k}}
\newcommand{\rtt}[2]{\vb{r}(t)=#1\,\vb{i}+#2\,\vb{j}}
\newcommand{\rs}[3]{\vb{r}(s)=#1\,\vb{i}+#2\,\vb{j}+#3\,\vb{k}}
\newcommand{\rss}[2]{\vb{r}(s)=#1\,\vb{i}+#2\,\vb{j}}
\newcommand{\vect}[4]{\vb{#1}=#2\,\vb{i}+#3\,\vb{j}+#4\,\vb{k}}
\newcommand{\vectt}[3]{\vb{#1}=#2\,\vb{i}+#3\,\vb{j}}


\DeclareMathOperator{\Arg}{Arg}



\begin{document}

\section{Απόσταση μεταξύ 2 σημείων}

\[
	d = \sqrt{{(x_{2} - x_{1})}^{2}- {(y_{2} - y_{1})}^{2}}  
\] 


\section{Κλίση $m$ ευθείας που διέρχεται από $ P_{1} (x_{1}, y_{1})$ και $ P_{2}
(x_{2}, y_{2})$}

\[
	m= \frac{y_{2} - y_{1}}{x_{2} - x_{1}} = \tan{\theta}	  
\] 

\section{Εξίσωση ευθείας που διέρχεται από σημεία $ P_{1} (x_{1}, y_{1}) $ και $
P_{2} (x_{2}, y_{2}) $}

\[
	y - y_{1} = m (x - x_{1}) \qq{ή} \frac{y - y_{1}}{x - x_{1}} =
	\frac{y_{2} - y_{1}}{x_{2} - x_{1}} 
\] 

\[
	 y = mx + b 
\]
όπου $ b = y_{1} - m x_{1} = \frac{x_{2} y_{1} - x_{1} y_{2}}{x_{2} - x_{1}}  $
 είναι το σημείο τομής της ευθείας με τον άξονα $ y $.

\section{Εξίσωση ευθείας που τέμνει τους άξονες $x$ και $y$ στα σημεία $ a\neq 0
$ και $ b\neq 0 $ αντίστοιχα}

\[
	\frac{x}{a} + \frac{y}{b} = 1 
\] 

\section{Κανονική μορφή εξίσωσης ευθείας}

\[
	x \cos{a} + y \sin{a} = p 
\] 
όπου $p$ είναι η απόσταση της αρχής των αξόνων $O$ από την ευθεία και $p$ η
γωνία που σχηματίζει η κάθετη απόσταση με τον θετικό ημιάξονα $x$.

\section{Γενική εξίσωση ευθεία}

\[
	 Ax + By + C = 0 
\] 

\section{Απόσταση σημείου $ P(x_{1}, y_{1}) $ από ευθεία $ Ax + By +C = 0 $ }

\[
	\frac{A x_{1} + B y_{1} + C}{\pm \sqrt{A^{2} + B^{2}}} 
\] 
όπου επιλέγεται το κατάλληλο πρόσημο ώστε η απόσταση να είναι μη αρνητική.

\section{Γωνία $ \phi $ μεταξύ ευθειών με κλίσεις $ m_{1} $, $ m_{2} $}

\[
	\tan{\phi} = \frac{m_{2} - m_{1}}{1 + m_{1} m_{2}}  
\] 
Οι ευθείες είναι παράλληλες αν και μόνον αν $ m_{1} = m_{2} $.
Οι ευθείες είναι κάθετες αν και μόνον αν $ m_{1} m_{2} = -1 $.

\section{Εμβαδόν Τριγώνου με κορυφές τα σημεία $ (x_{1}, y_{1}) $, $ (x_{2},
y_{2}) $ και $ (x_{3}, y_{3}) $}

\[
	E = \pm \frac{1}{2} \begin{vmatrix}
		x_{1} & x_{1} & 1 \\
		x_{2} & y_{2} & 1 \\
		x_{3} & y_{3} & 1
	\end{vmatrix} = \pm \frac{1}{2} (x_{1} y_{2} + y_{1} x_{3} - y_{2} x_{3} - y
_{1} x_{2} - x_{1} y_{3})
\] 
όπου επιλέγεται το κατάλληλο πρόσημο ώστε το εμβαδό να είναι μη αρνητικό.
Αν το εμβαδό είναι μηδέν, τότε όλα τα σημεία βρίσκονται πάνω στην ίδια ευθεία.

\section{Μετασχηματισμοί συντεταγμένων}

\subsection{Μεταφορά}

\[
	\begin{cases}
		x = x' + x_{0} \\
		y = y' + y_{0} 
	\end{cases}
	\qq{ή}
	\begin{cases}
		x' = x - x_{0} \\
		y' = y - y_{0} 
	\end{cases}
\] 
όπου $ (x,y) $ και $ (x',y') $ οι συντεταγμένες ως προς τα συστήματα $ xy
$ και $ x'y' $ αντίστοιχα και $ (x_{0}, y_{0}) $ οι συντεταγμένες της αρχής $
O' $ ως προς τη σύστημα $ xy $.

\subsection{Περιστροφή}

\[
	 \begin{cases}
		 x = x' \cos{a} - y' \sin{a} \\
		 y = x' \sin{a} + y' \cos{a} 
	 \end{cases}
	 \qq{ή} 
	 \begin{cases}
		 x' = x \cos{a} + y \sin{a} \\
		 y' = y \cos{a} - x \sin{a}
	 \end{cases}
\]

	 όπου οι αρχές των δύο συστημάτων $ xy $ και $ x'y' $ ταυτίζονται, αλλά ο
	 άξονας $ x' $ σχηματίζει γωνία $ a $ με τον θετικό ημιάξονα $x$.

	 \subsection{Μεταφορά και Περιστροφή}

	 \[
		  \begin{cases}
			  x = x' \cos{a} - y' \sin{a} + x_{0} \\
			  y = x' \sin{a} + y' \cos{a} + y_{0}
		  \end{cases} 
		  \qq{ή}
		  \begin{cases}
			  x' = (x-x_{0}) \cos{a} + (y - y_{0}) \sin{a} \\
			  y' - (y-y_{0}) \cos{a} - (x - x_{0}) \sin{a} 
		  \end{cases}
	 \] 
	 όπου η αρχή $ O' $ του συστήματος $ x'y' $ έχει συντεταγμένες $ (x_{0},
	 y_{0}) $ ως προς το σύστημα $ xy $ και ο άξονας $ x' $ σχηματίζει γωνία $a$
	 με τον θετιό ημιάξονα. 
	






\end{document}
