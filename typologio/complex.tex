
\everymath{\displaystyle}


\section{Μιγαδικοί Αριθμοί}

\subsection*{Ορισμός}

  Έστω $a,b \in \mathbb{R}$ και $i=\sqrt{-1}$ η
  \textbf{φανταστική μονάδα}. Κάθε
  παράσταση της μορφής $a+bi$ παριστάνει έναν {\color{blue} μιγαδικό αριθμό}.
  Με  $\Re z$ και $\Im z$ συμβολίζουμε το {\color{blue} πραγματικό} και το
  {\color{blue} φανταστικό} μέρος αντίστοιχα, του μιγαδικού αριθμού $z=a+bi$.
  Ισχύει,  $\Re z=a$ και $\Im z=b$.
  Το σύνολο των μιγαδικών αριθμών το συμβολίζουμε με $\mathbb{C}$.


  \subsection*{Ισότητα}

   Έστω $z_{1}=a+bi$ και $z_{2}=c+di$ μιγαδικοί αριθμοί.
  \[
    z_{1}=z_{2}\Leftrightarrow a=c \quad \text{και} \quad b=d
  \]


  \subsection*{Γεωμετρική Αναπαράσταση των Μιγαδικών Αριθμών}

  Έστω $xOy$ καρτεσιανό σύστημα αξόνων και $z=a+bi$ μιγαδικός αριθμός. Στον μιγαδικό αριθμό $z$  αντιστοιχίζουμε το ζεύγος $(a,b)$ και επομένως το σημείο $M(a,b)$ με συντεταγμένες $a,b$ το οποίο ονομάζουμε \textcolor{blue}{γεωμετρική εικόνα} του μιγαδικού $z$.

  Επίσης στο σημείο $M(a,b)$ και άρα και στον μιγαδικό αριθμό $z$ αντιστοιχίζουμε το διάνυσμα θέσης $\vec{OM}$.

  \subsubsection*{Μέτρο}

  Το {\color{blue} μέτρο} του μιγαδικού αριθμού $z=a+bi$ το
  συμβολίζουμε με $\abs{z}$ και ισχύει
\[
  \abs{z}=\sqrt{a^2+b^2}.
\]

  \begin{itemize}
    \item   $OM=\sqrt{(a^{2}+b^{2})}=\abs{z}$
    \item   $\abs{a}=\abs{\Re z}\leq \abs{z}$
    \item   $\abs{b}=\abs{\Im z}\leq \abs{z}$
  \end{itemize}

  \subsection*{Πράξεις}

    Έστω $z_{1}=a+bi$ και $z_{2}=c+di$ μιγαδικοί αριθμοί.

\begin{multicols}{2}
  \subsubsection*{Πρόσθεση}
  $z_{1}+z_{2}=(a+bi)+(c+di)=(a+c)+(b+d)i$

  \columnbreak

  \subsubsection*{Αφαίρεση}
  $z_{1}-z_{2}=(a+bi)-(c+di)=(a-c)+(b-d)i$
\end{multicols}

\begin{multicols}{2}
  \subsubsection*{Πολλαπλασιασμός}
        \begin{align*}
          z_{1}\cdot z_{2}&=(a+bi)\cdot (c+di) \\
            &=ac+adi+bci+bdi^{2} \\
            &=ac-bd+(ad+bc)i \\
        \end{align*}

        \columnbreak

        \subsubsection*{Διαίρεση}
        \begin{align*}
          \frac{z_{1}}{z_{2}}&=\frac{(a+bi)}{(c+di)}     =\frac{(a+bi)(c-di)}{(c+di)(c-di)} \\
            &=\frac{ac+bd+(-ad+bc)}{b^{2}+d^{2}} \\
            &=\frac{ac+bd}{c^2+d^2}+\frac{bc-ad}{c^2+d^2}
        \end{align*}

\end{multicols}



  \subsubsection*{Συνοπτικά}
  \begin{itemize}
    \item $z_{1}+z_{2}=(a+c)+(b+d)i$
    \item $z_{1}-z_{2}=(a-c)+(b-d)i$
    \item $z_{1}\cdot z_{2}=(ac-bd)+(ad+bc)i$
    \item $\frac{z_{1}}{z_{2}}=\frac{ac+bd}{c^2+d^2}+\frac{bc-ad}{c^2+d^2}i$
  \end{itemize}

\subsection*{Ιδιότητες των Πράξεων}

Έστω $z_{1}, z_{2}, z_{3}$ και $z$ μιγαδικοί
    αριθμοί.

\begin{table}[h!]
  \centering
  \begin{tabular}{cc}
    \toprule \\
    Πρόσθεση & Πολλαπλασιασμός \\
    \midrule \\
    $z_{1}+z_{2}=z_{2}+z_{1}$ & $z_{1}\cdot z_{2}=z_{2}\cdot z_{1}$ \\
    $z_{1}+(z_{2}+z_{3})=(z_{1}+z_{2})+z_{3}$ & $z_{1}\cdot (z_{2}\cdot z_{3})=(z_{1}\cdot z_{2})\cdot z_{3}$ \\
    $z+0=0+z=z$ &  $z\cdot 1=1\cdot z=z$ \\
    $z+(-z)=(-z)+z=0$ & $z\cdot z^{-1}=z^{-1}\cdot z=1$ \\
    \multicolumn{2}{c}{
    $z_{1}\cdot (z_{2}+z_{3})=z_{1}\cdot z_{2}+z_{1}\cdot z_{3}$} \\
    \bottomrule
  \end{tabular}
\end{table}






$z_{1}\cdot (z_{2}+z_{3})=z_{1}\cdot z_{2}+z_{1}\cdot z_{3}$




  \subsection*{Δυνάμεις}

   Έστω $z\in \mathbb{C}$, $n\in\mathbb{Z}$.
    \begin{itemize}
      \item $z^{n}=\underbrace{z\cdot z \cdots z}_{n\; \text{φορές}}$, $n>0$
      \item $z^{0}=1$
      \item $z^{n}=(z^{-1})^{-n}$, $n<0$
    \end{itemize}

\subsubsection*{Ιδιότητες των Δυνάμεων}

     Έστω $z, z_{1}, z_{2}\in \mathbb{C}$, $n\in\mathbb{Z}$.
    \begin{itemize}
      \item $z^{m}\cdot  z^{n}=z^{m+n}$
      \item $\frac{z^{m}}{z^{n}}=z^{m-n}$
      \item $(z^{m})^{n}=z^{m\cdot n}$
      \item $(z_{1}\cdot z_{2})^{n}=z_{1}^{n}\cdot z_{2}^{n}$
      \item $(\frac{z_{1}}{z_{2}})^{n}=\frac{z_{1}^{n}}{z_{2}^{n}}$
      \item $z=0\Rightarrow 0^{n}=0$, $n>0$
    \end{itemize}

    \subsubsection*{Δυνάμεις της Φανταστικής Μονάδας}

          \begin{multicols}{2}
        \begin{itemize}
        \item $i^{0}=1$
        \item $i^{1}=i$
        \item $i^{2}=-1$
        \item $i^{3}=i^{2}\cdot i=-i$
\end{itemize}

        \columnbreak

  \begin{itemize}
        \item $i^{4}=i^{3}\cdot i=1$
        \item $i^{5}=i^{4}\cdot i=i$
        \item $i^{6}=i^{5}\cdot i=-1$
        \item $i^{7}=i^{6}\cdot i=-i$
  \end{itemize}
    \end{multicols}


        Αποδεικνύεται με επαγωγή ότι για κάθε $n>0$

        \begin{multicols}{2}
          \begin{itemize}
            \item $i^{4n}=1$
            \item $i^{4n+1}=i$
            \item $i^{4n+2}=-1$
            \item $i^{4n+3}=-i$
          \end{itemize}
          \columnbreak
          \begin{itemize}
            \item $i^{n}=(i^{-1})^{-n}=\left(\frac{1}{i}\right)^{-n}=(-i)^{-n}$, $n<0$
          \end{itemize}
        \end{multicols}

\subsection*{Συζυγής}

Έστω $z=a+bi$ μιγαδικός αριθμός. Με $\overline{z}$ συμβολίζουμε
 τον {\color{blue} συζυγή} μιγαδικό αριθμό του $z$ και ισχύει $\overline{z}=a-bi$.

\subsubsection*{Ιδιότητες Συζυγών}


\begin{itemize}
  \item $\overline{\overline{z}}=z$
  \item $\overline{z_{1}+z_{2}}=\overline{z_{1}}+\overline{z_{2}}$ και
  \item $\overline{z_{1}\cdot z_{2}}=\overline{z_{1}}\cdot\overline{z_{2}}$ και
  \item $\overline{\left(\frac{z_{1}}{z_{2}}\right)}=\frac{\overline{z_{1}}}{\overline{z_{2}}}$, $z_{2}\neq 0$
  \item $\overline{z^{n}}=\overline{z}^n$
  \item $z\cdot \overline{z}=a^{2}+b^{2}$
  \item $z=\overline{z}\Leftrightarrow z\in\mathbb{R}$
  \item $\overline{\left(\sum_{k=1}^{n}z_{k}\right)}=\sum_{k=1}^{n}\overline{z}_{k}\Leftrightarrow \overline{z_{1}+z_{2}+\cdots +z_{n}}=\overline{z_{1}}+\overline{z_{2}}+\cdots \overline{z_{n}}$
  \item $\overline{\left(\prod_{k=1}^{n}z_{k}\right)}=\prod_{k=1}^{n}\overline{z}_{k}\Leftrightarrow \overline{z_{1}\cdot z_{2}\cdots z_{n}}=\overline{z_{1}}\cdot \overline{z_{2}}\cdots \overline{z_{n}}$
\end{itemize}

  \begin{itemize}


    \item $\Re z=\frac{z+\overline{z}}{2}$
    \item $\Im z=\frac{z-\overline{z}}{2i}$


  \end{itemize}



  \subsection*{Ιδιότητες Μέτρου}

\begin{itemize}
  \item []
  \item $\abs{z}\geq 0$ και $\abs{z}=0 \Leftrightarrow z=0$
  \item $-\abs{z}\leq \Re z \leq \abs{z}$
  \item $-\overline{z}\leq \Re z \leq \abs{\Re z}\leq \overline{z}$
  \item $-\overline{z}\leq \Im z \leq \abs{\Im z}\leq \overline{z}$
  \item $\abs{z}=\abs{-z}=\abs{\overline{z}}=\abs{-\overline{z}}$
  \item $z\cdot\overline{z}=\abs{z}^{2}\in\mathbb{R}$
  \item $\abs{z_{1}\cdot z_{2}}=\abs{z_{1}}\cdot \abs{z_{2}}$
  \item $\abs{z_{1}-z_{2}}\leq\abs{z_{1}\pm z_{2}}\leq \abs{z_{1}}+\abs{z_{2}}$
  \item $\abs{\frac{z_{1}}{z_{2}}}=\frac{\abs{z_{1}}}{\abs{z_{2}}}$, $z_{2}\neq 0$
  \item $\abs{\prod_{k=1}^{n}z_{k}}=\prod_{k=1}^{n}\abs{z_{k}}$
  \item $\abs{\sum_{k=1}^{n}z_{k}}\leq \sum_{k=1}^{n}\abs{z_{k}}$
  \item $\abs{z^{k}}=\abs{z}^{k}$, $k\in\mathbb{Z}$, άρα $\abs{z^{-1}}=\abs{z}^{-1}$, $z\neq 0$
  \item $\abs{z_{1}+z_{2}}^{2}+\abs{z_{1}-z_{2}}^{2}=2(\abs{z_{1}}^{2}+\abs{z_{2}}^{2})$
  \item $\abs{z+1}\geq\frac{1}{\sqrt{2}}$ και $\abs{z^{2}+1}\geq 1$
  \item $\abs{z_{1}+z_{2}}^{2}+\abs{z_{2}+z_{3}}^{2}+\abs{z_{3}+z_{1}}^{2}=\abs{z_{1}}^{2}+\abs{z_{2}}^{2}+\abs{z_{3}}^{2}+\abs{z_{1}+z_{2}+z_{3}}^{2}$
  \item $\abs{1+z_{1}\overline{z_{2}}}^{2}\pm\abs{z_{1}-z_{2}}^{2}=(1\pm\abs{z_{1}}^{2})(1\pm\abs{z_{2}}^{2})$
  \item $\abs{z_{1}+z_{2}+z_{3}}^{2}+\abs{-z_{1}+z_{2}+z_{3}}^{2}+\abs{z_{1}-z_{2}+z_{3}}^{2}+\abs{z_{1}+z_{2}-z_{3}}^{2}=4(\abs{z_{1}}^{2}+\abs{z_{2}}^{2}+\abs{z_{3}}^{2})$
\end{itemize}

\subsection*{Πολική μορφή Μιγαδικού αριθμού}

Έστω $z=a+bi$ μιγαδικός αριθμός. Έστω $r=\abs{z}$ και $\theta$ η γωνία που σχηματίζει το διάνυσμα θέσης του $z$ με τον θετικό ημιάξονα $x$ και η οποία μετριέται σε ακτίνια. Τότε $x=r\cos \theta$ και $y=r\sin \theta$ και επομένως ο μιγαδικός $z$ μπορεί να γραφεί σε \textcolor{blue}{τριγωνομετρική} ή \textcolor{blue}{πολική} μορφή ως:
\[
z=r(\cos\theta+i\sin\theta)
\]
και χρησιμοποιώντας την ταυτότητα \textlatin{Euler} $e^{ix}=\cos x+i\sin x$, $x\in\mathbb{R}$ σε \textcolor{blue}{εκθετική} μορφή ως:
\[
    z= re^{i\theta}
\]

Oι τιμές των $r, \theta$ λέγονται \textcolor{blue}{πολικές συντεταγμένες} του $z$. Κάθε τιμή του $\theta$ λέγεται \textcolor{blue}{όρισμα} του $z$, και με $\arg z$ συμβολίζουμε το σύνολο των ορισμάτων του $z$, που είναι άπειρα και διαφέρουν μεταξύ τους κατά $2\pi$. Αν $z=0$ τότε δεν ορίζεται το όρισμα του. Ισχύει:
\[
\tan\theta=\frac{y}{x}
\]

Το \textcolor{blue}{Πρωτεύον όρισμα} του $\arg z$ το συμβολίζουμε με $\Arg z$ και είναι η μοναδική εκείνη τιμή του $\theta\in \arg z$ ώστε $0\leq\theta<2\pi$. Ισχύει:
\[
\arg z=\Arg z+2k\pi, \quad k\in\mathbb{Z}
\]

Από τον ορισμό του $\Arg z$ προκύπτει ότι αν $z=a+bi$, τότε:
\begin{enumerate}
  \item Αν $a\neq 0$ τότε από τη σχέση $\tan\theta=\frac{y}{x}$ έχουμε:
  \[
\Arg z=\arctan\frac{b}{a}+k\pi, \quad \text{όπου}\quad k=\begin{cases}
  0, & \text{για}\quad a>0, b\geq 0 \\
  1, & \text{για}\quad a<0 \\
  2, & \text{για}\quad a>0, b<0
\end{cases}
  \]

  \item Αν $a=0$ και $b\neq 0$ τότε
  \[
\Arg=\begin{cases}
\sfrac{\pi}{2},  &\text{για}\quad   b>0 \\
\sfrac{3\pi}{2}, &\text{για}\quad   b<0
\end{cases}
  \]
\end{enumerate}

Έστω $z_{1}=r_{1}(\cos\theta_{1}+i\sin\theta_{1}), z_{2}=r_{2}(\cos\theta_{2}+i\sin\theta_{2})$ και $z=r(\cos\theta+i\sin\theta)$

\subsubsection*{Ισότητα σε Πολική μορφή}

\[
z_{1}=z_{2}\Leftrightarrow r_{1}=r_{2} \quad\text{και}\quad \Arg z_{1}=\Arg z_{2}\quad \text{ή}\quad \theta_{1}=\theta_{2}+2k\pi,\quad k\in\mathbb{Z}
\]

\subsubsection*{Πράξεις σε Πολική μορφή}

\begin{table}[h!]
  \centering
  \begin{tabular}{cc}
    \toprule \\
    Τριγωνομετρική μορφή & Εκθετική μορφή \\
    \midrule
    $z_{1}\cdot z_{2}=r_{1}r_{2}[\cos(\theta_{1}+\theta_{2})+i(\sin\theta_{1}+\theta_{2})]$ & $z_{1}\cdot z_{2}=r_{1}r_{2}e^{i(\theta_{1}+\theta_{2})}$ \\
    $\frac{z_{1}}{z_{2}}=\frac{r_{1}}{r_{2}}[(\cos(\theta_{1}-\theta_{2})+i(\sin\theta_{1}-\theta_{2}))]$ & $\frac{z_{1}}{z_{2}}=\frac{r_{1}}{r_{2}}e^{i(\theta_{1}-\theta_{2})}$ \\
    $z^{-1}=\frac{1}{z}=\frac{1}{r}(\cos\theta-i\sin\theta)$ & $z^{-1}=\frac{1}{r}e^{-i\theta}$ \\
    $\overline{z}=r(\cos\theta-i\sin\theta)$ & $\overline{z}=re^{-i\theta}$ \\
    \bottomrule
  \end{tabular}
\end{table}



\subsubsection*{Θεώρημα \textlatin{De Moivre}}

    Αν $z=r(\cos\theta+i\sin\theta)$ και $n\in\mathbb{N}$ τότε
    \[
    z^{n}=r^{n}(\cos n\theta+i\sin n\theta)
    \]

\subsubsection*{Ρίζες Μιγαδικού αριθμού}

Κάθε μιγαδικός αριθμός της μορφής $a=r(\cos\theta+i\sin\theta), a\neq 0$ έχει $n$ ακριβώς διαφορετικές $n$-οστές ρίζες, που δίνονται από τον τύπο:
\[
z_{k}=\sqrt[n]{r}\left(\cos\frac{\theta+2k\pi}{n}+i\sin\frac{\theta+2k\pi}{n}\right), \quad k=0,1,\ldots, n-1
\]


