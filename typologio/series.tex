\documentclass[a4paper,10pt]{report}
\usepackage{etex}
%%%%%%%%%%%%%%%%%%%%%%%%%%%%%%%%%%%%%%
% Babel language package
\usepackage[english,greek]{babel}
% Inputenc font encoding
\usepackage[utf8]{inputenc}
%%%%%%%%%%%%%%%%%%%%%%%%%%%%%%%%%%%%%%

\usepackage{extsizes}
\usepackage{multicol}

%%%%% math packages %%%%%%%%%%%%%%%%%%
\usepackage[intlimits]{amsmath}
\usepackage{amssymb}
\usepackage{amsfonts}
\usepackage{amsthm}
\usepackage{proof}
\usepackage{mathtools}

\usepackage[italicdiff]{physics}
\usepackage{siunitx}
\usepackage{xfrac}

%%%%%%% symbols packages %%%%%%%%%%%%%%
\usepackage{bm} %for use \bm instead \boldsymbol in math mode
\usepackage{dsfont}
\usepackage{stmaryrd}
%%%%%%%%%%%%%%%%%%%%%%%%%%%%%%%%%%%%%%%


%%%%%% graphics %%%%%%%%%%%%%%%%%%%%%%%
\usepackage{graphicx}
\usepackage{color}
%\usepackage{xypic}
%\usepackage[all]{xy}
%\usepackage{calc}

%%%%%% tables %%%%%%%%%%%%%%%%%%%%%%%%%
\usepackage{array}
\usepackage{booktabs}
\usepackage{multirow}
\usepackage{makecell}
\usepackage{minibox}
%%%%%%%%%%%%%%%%%%%%%%%%%%%%%%%%%%%%%%%

\usepackage{enumerate}

\usepackage{fancyhdr}
%%%%% header and footer rule %%%%%%%%%
\setlength{\headheight}{14pt}
\renewcommand{\headrulewidth}{0pt}
\renewcommand{\footrulewidth}{0pt}
\fancypagestyle{plain}{\fancyhf{}
\fancyhead{}
\lfoot{}
\rfoot{\small \thepage}}
\fancypagestyle{vangelis}{\fancyhf{}
\rhead{\small \leftmark}
\lhead{\small }
\lfoot{}
\rfoot{\small \thepage}}
%%%%%%%%%%%%%%%%%%%%%%%%%%%%%%%%%%%%%%%

\usepackage{hyperref}
\usepackage{url}
%%%%%%% hyperref settings %%%%%%%%%%%%
\hypersetup{pdfpagemode=UseOutlines,hidelinks,
bookmarksopen=true,
pdfdisplaydoctitle=true,
pdfstartview=Fit,
unicode=true,
pdfpagelayout=OneColumn,
}
%%%%%%%%%%%%%%%%%%%%%%%%%%%%%%%%%%%%%%

\usepackage[space]{grffile}

\usepackage{geometry}
\geometry{left=25.63mm,right=25.63mm,top=36.25mm,bottom=36.25mm,footskip=24.16mm,headsep=24.16mm}

%\usepackage[explicit]{titlesec}
%%%%%% titlesec settings %%%%%%%%%%%%%
%\titleformat{\chapter}[block]{\LARGE\sc\bfseries}{\thechapter.}{1ex}{#1}
%\titlespacing*{\chapter}{0cm}{0cm}{36pt}[0ex]
%\titleformat{\section}[block]{\Large\bfseries}{\thesection.}{1ex}{#1}
%\titlespacing*{\section}{0cm}{34.56pt}{17.28pt}[0ex]
%\titleformat{\subsection}[block]{\large\bfseries{\thesubsection.}{1ex}{#1}
%\titlespacing*{\subsection}{0pt}{28.80pt}{14.40pt}[0ex]
%%%%%%%%%%%%%%%%%%%%%%%%%%%%%%%%%%%%%%

%%%%%%%%% My Theorems %%%%%%%%%%%%%%%%%%
\newtheorem{thm}{Θεώρημα}[section]
\newtheorem{cor}[thm]{Πόρισμα}
\newtheorem{lem}[thm]{λήμμα}
\theoremstyle{definition}
\newtheorem{dfn}{Ορισμός}[section]
\newtheorem{dfns}[dfn]{Ορισμοί}
\newtheorem{ex}[thm]{Παραδειγμα}
\theoremstyle{remark}
\newtheorem{remark}{Παρατήρηση}[section]
\newtheorem{remarks}[remark]{Παρατηρήσεις}
%%%%%%%%%%%%%%%%%%%%%%%%%%%%%%%%%%%%%%%

%%%%%%% nesting newcommands $$$$$$$$$$$$$$$$$$$
\newcommand{\function}[1]{\newcommand{\nvec}[2]{#1(##1_1,\ldots, ##1_##2)}}

\newcommand{\linode}[2]{#1_n(x)#2^{(n)}+#1_{n-1}(x)#2^{(n-1)}+\cdots +#1_0(x)#2=g(x)}

\newcommand{\vecoffun}[3]{#1_0(#2),\ldots ,#1_#3(#2)}

\newcommand{\mysum}[1]{\sum_{n=#1}^{\infty}}



\renewcommand{\vector}[1]{(x_1,x_2,\ldots,x_{#1})}
\newcommand{\avector}[2]{(#1_1,#1_2,\ldots,#1_{#2})}
\newcommand{\aDEFvector}[2][a]{(#1_1,#1_2,\ldots,#1_{#2})}

\newcommand{\rt}[3]{\vb{r}(t)=#1\,\vb{i}+#2\,\vb{j}+#3\,\vb{k}}
\newcommand{\rtt}[2]{\vb{r}(t)=#1\,\vb{i}+#2\,\vb{j}}
\newcommand{\rs}[3]{\vb{r}(s)=#1\,\vb{i}+#2\,\vb{j}+#3\,\vb{k}}
\newcommand{\rss}[2]{\vb{r}(s)=#1\,\vb{i}+#2\,\vb{j}}
\newcommand{\vect}[4]{\vb{#1}=#2\,\vb{i}+#3\,\vb{j}+#4\,\vb{k}}
\newcommand{\vectt}[3]{\vb{#1}=#2\,\vb{i}+#3\,\vb{j}}


\DeclareMathOperator{\Arg}{Arg}



\begin{document}

\section{Αριθμητικές Σειρές}

\subsection{Αριθμητική Πρόοδος}

$ a + (a+d) + (a+2d) + \cdots + (a+(n-1)d) = \frac{1}{2} n[2a + (n-1)d] =
\frac{1}{2} n(a+1) $ 
όπου $l = a + (n-1)d$ είναι ο τελευταίος όρος.

Μερικές ειδικές περιπτώσεις είναι
$ 1 + 2 + 3 + \cdots + n = \frac{1}{2} n(n+1) $ 
$ 1 + 3 + 5 + \cdots + (2n-1) = n^{2} $ 

\subsection{Γεωμετρική Σειρά}

$ a + ar + ar^{2} + ar^{3} + \cdots + ar^{n-1} = \frac{a(1-r^{n})}{1-r} =
\frac{a-rl}{1-r}$
όπου $ l=ar^{n-1} $ είναι ο τελευταίος όρος και $ r\neq 1 $. 

Αν $ -1<r<1 $, τότε $ a + ar + ar^{2} + ar^{3} + \cdots = \frac{a}{1-r}
$. 

\subsection{Αριθμητικο-Γεωμετρική Σειρά}
$ a + (a+d)r + (a+2d)r^{2} + \cdots + [a+(n-1)r^{n-1}] = \frac{a(1-r^{n})}{1-r}
= \frac{a-rl}{1-r} $ 
όπου $ r\neq 1 $, τότε
$ a + (a+d)r + (a+2d)r^{2} + \cdots = \frac{a}{1-r} + \frac{rd}{(1-r)^{2}}$. 

\subsection{Αθροίσματα Δυνάμεων θετικών ακεραίων}
$ 1^{p} + 2^{p} + 3^{p} + \cdots + n^{p} = \frac{n^{p+1}}{p+1} +
\frac{1}{2} n^{p} + \frac{B_{1} p n^{p-1}}{2!} - \frac{B_{2}
p(p-1)(p-2)n^{p-3}}{4!} + \cdots $
όπου η σειρά τερματίζει στο $ n^{2} $ ή στο $ n $ αναλόγως αν ο $ p $ είναι
περιττός ή άρτιος, και $ B_{k} $ είναι οι αριθμοί \textlatin{Bernoulli}.

Μερικές ειδικές περιπτώσεις είναι
\begin{tabular}{l}
    $1 + 2 + 3 + \cdots + n = \frac{n(n+1)}{2} $ \\ 
    $1^{2} + 2^{2} + 3^{2} + \cdots + n^{2} = \frac{n(n+1)(2n+1)}{6} $ \\
    $1^{3} + 2^{3} + 3^{3} + \cdots + n^{3} = \frac{n^{2}(n+1)^{2}}{4} =
    (1+2+3+\cdots + n)^{2}$ \\
    $ 1^{4} + 2^{4} + 3^{4} + \cdots + n^{4} =
    \frac{n(n+1)(2n+1)(3n^{2}+3n-1)}{30}$ \\
\end{tabular} 

Αν $ S_{k} = 1^{k} + 2^{k} + 3^{k} + \cdots + n^{k} $ όπου $ k $ και $ n
$ είναι θετικοί ακέραιοι, τότε
\[
    \binom{k+1}{1} S_{1} + \binom{k+1}{2} S_{2} + \cdots + \binom{k+1}{k}
    S_{k} = (n+1)^{k+1} - (n+1).
\] 

\subsection{Σειρές αντίστροφων δυνάμεων θετικών ακεραίων}

\begin{tabular}{l}
    $ 1 - \frac{1}{2} + \frac{1}{3} - \frac{1}{4} + \frac{1}{5} - \cdots =
    \ln{2} $ \\
    $ 1 - \frac{1}{3} + \frac{1}{5} - \frac{1}{7} + \frac{1}{9} - \cdots =
    \frac{\pi}{4} $ \\
    $ 1 - \frac{1}{4} + \frac{1}{7} - \frac{1}{10} + \frac{1}{13} - \cdots =
    \frac{\pi \sqrt{3}}{9} + \frac{1}{3} \ln{2} $ \\
    $ 1 - \frac{1}{5} + \frac{1}{9} - \frac{1}{13} + \frac{1}{17} - \cdots =
    \frac{\pi \sqrt{2}}{8} + \frac{\sqrt{2} \ln{(1 + \sqrt{2})})}{4} $ \\
    $ \frac{1}{2} - \frac{1}{5} + \frac{1}{8} - \frac{1}{11} + \frac{1}{14}
    - \cdots = \frac{\pi \sqrt{3}}{9} + \frac{1}{3} \ln{2}$ \\
    $ \frac{1}{1^{2}} + \frac{1}{2^{2}} + \frac{1}{3^{2}} + \frac{1}{4^{2}}
    + \cdots = \frac{\pi ^{2}}{6} $ \\
    $ \frac{1}{1^{4}} + \frac{1}{2^{4}} + \frac{1}{3^{4}} + \frac{1}{4^{4}}
    + \cdots = \frac{\pi^{4}}{90}  $ \\
    $ \frac{1}{1^{6}} + \frac{1}{2^{6}} + \frac{1}{3^{6}} + \frac{1}{4^{6}}
    + \cdots = \frac{\pi ^{6}}{945} $ \\
    $ \frac{1}{1^{2}} - \frac{1}{2^{2}} + \frac{1}{3^{2}} - \frac{1}{4^{2}}
    + \cdots = \frac{\pi ^{2}}{12} $ \\
    $ \frac{1}{1^{4}} - \frac{1}{2^{4}} + \frac{1}{3^{4}} - \frac{1}{4^{4}}
    + \cdots = \frac{7 \pi ^{4}}{720} $ \\
    $ \frac{1}{1^{6}} - \frac{1}{2^{6}} + \frac{1}{3^{6}} - \frac{1}{4^{6}}
    + \cdots = \frac{31 \pi ^{6}}{30240} $ \\
    $ \frac{1}{1^{2}} + \frac{1}{3^{2}} + \frac{1}{5^{2}} + \frac{1}{7^{2}}
    + \cdots = \frac{\pi ^{2}}{8} $ \\
    $ \frac{1}{1^{4}} + \frac{1}{3^{4}} + \frac{1}{5^{4}} + \frac{1}{7^{4}}
    + \cdots = \frac{\pi ^{4}}{96}  $ \\
    $ \frac{1}{1^{6}} + \frac{1}{3^{6}} + \frac{1}{5^{6}} + \frac{1}{7^{6}} +
    \cdots = \frac{\pi ^{6}}{960} $ \\
    $ \frac{1}{1^{3}} - \frac{1}{3^{3}} + \frac{1}{5^{3}} - \frac{1}{7^{3}} +
    \cdots = \frac{\pi ^{3}}{32} $ \\
    $ \frac{1}{3^{3}} + \frac{1}{3^{3}} - \frac{1}{5^{3}} + \frac{1}{7^{3}}
    + \cdots = \frac{3 \pi ^{3}\sqrt{2}}{128} $ \\
    $ \frac{1}{1 \cdot 3} + \frac{1}{3 \cdot 5} + \frac{1}{5 \cdot 7} +
    \frac{1}{7 \cdot 9} + \cdots = \frac{1}{2} $ \\
    $ \frac{1}{1 \cdot 3} + \frac{1}{2 \cdot 4} + \frac{1}{3 \cdot 5} +
    \frac{1}{4 \cdot 6} + \cdots = \frac{3}{4} $ \\
    $ \frac{1}{1^{2} \cdot 3^{2}} + \frac{1}{3^{2} \cdot 5^{2}} +
    \frac{1}{5^{2} \cdot 7^{2}} + \frac{1}{7^{2} \cdot 9^{2}} =
    \frac{\pi ^{2} - 8}{16}   $ \\
    $ \frac{1}{1^{2} \cdot 2^{2} \cdot 3^{2}} + \frac{1}{2^{2} \cdot 3^{2} \cdot
    4^{2}} + \frac{1}{3^{2} \cdot 4^{2} \cdot 5^{2}} + \cdots =
    \frac{4 \pi ^{2} - 39}{16} $ \\
\end{tabular}


% TODO check the other series with u, B_p, E_p ... 

\subsection{Αλλες Σειρές}
\begin{tabular}{l}
$ \frac{1}{2} + \cos a + \cos{2a} + \cdots + \cos{na} = \frac{\sin{(n+
\frac{1}{2})}a}{2 \sin{(\frac{a}{2})}}$ \\
$ \sin{a} + \sin{2a} + \sin{3a} + \cdots + \sin{na} =
\frac{\sin{[\frac{1}{2} (n+1)]a}  \sin{\frac{1}{2} n a}
}{\sin{\frac{a}{2}}}  $ \\
$ 1 + r \cos{a} + r^{2} \cos{2a} + r^{3} \cos{3a} + \cdots = \frac{1 - r
\cos{a}}{1 - 2r \cos{a} + r^{2}}, \abs{r}<1$ \\
$ r \sin{a} + r^{2} \sin{2a} + r^{3} \sin{3a} + \cdots = \frac{r \sin{a}}
{1 - 2r \cos{a} + r^{2}}, \abs{r}<1$ \\
$ 1 + r \cos{a} + r^{2} \cos{2a} + \cdots + r^{n} \cos{na} = \frac{r^{n+2}
\cos{na} - r^{n+1} \cos{(n+1)a - r \cos{a} + 1}}{1 - 2a \cos{a} + r^{2}}$ \\
$ r \sin{a} + r^{2} \sin{2a} + \cdots r^{n} \sin{na} = \frac{r \sin{a} - r^{n+1}
\sin{(n+1)a} + r^{n+2} \sin{na}}{1 - 2r \cos{a} + r^{2}} $
\end{tabular}

%TODO check other summation formulas


\section{Σειρές \textlatin{Taylor}}

\[
    f(x) = f(a) + f'(a)(x-a) + \frac{f''(x)}{2!} (x-a)^{2} + \cdots +
    \frac{f^{(n-1)}}{(n-1)!} (x-a)^{n-1} + R_{n}
\] 
όπου $ R_{n} $ το υπόλοιπο μετά από $n $ όρους, δίνεται από έναν από τους δύο
παρακάτω τύπους:

\textbf{\textlatin{Lagrange's Form:}} $ R_{n} = \frac{f^{n(\xi)}}{n!} (x-a)^{n} $

\textbf{\textlatin{Cauchy's Form:}} $ R_{n} = \frac{f^{n}(\xi)}{(n-1)!}
(x-\xi)^{n-1}(x-a) $

Η τιμή $\xi$ η οποία μπορεί να είναι διαφορετική στους 2 τύπους, βρίσκεται
μεταξύ των $ a $ και $ x$. Ο παραπάνω τύπος ισχύει αν η $ f(x) $ έχει συνεχείς
παραγώγους, τουλάχιστον $ n $-οστής τάξης. 

Αν $ \lim_{x\to \infty} R_{n} = 0 $, η άπειρη σειρά η οποία προκύπτει ονομάζεται
σειρά \textlatin{Taylor} της $ f(x) $ για $ x=a $. Αν $ a = 0 $, τότε η σειρά
ονομάζεται σειρά \textlatin{Maclaurin}. Αυτές οι σειρές, γνωστές και ως
δυναμοσειρές, συγκλίνουν για κάθε τιμή του $x$ σε κάποιο διάστημα, το οποίο
ονομάζεται διάστημα σύγκλισης και αποκλίνουν για κάθε τιμή του $x$ εκτός αυτού
του διαστήματος.

\section{Διωνυμικές Σειρές}

$ (a+x) $<++>





































\end{document}

